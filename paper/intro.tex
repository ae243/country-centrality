\section{Introduction}
\label{intro}

Internet routing is usually studied and decided at the AS (Autonomous System) level.  An AS typically controls traffic as it transits the internal network and then defines their own policies for filtering, monitoring, and routing traffic to the next AS.  It is becoming increasingly common for ASes to follow the policies of their respective governments, which could include censorship or surveillance.  Once Internet traffic enters a country, it is subject to that country's policies.  More and more countries are either tyring or have already passed laws that allow for mass surveillance on their citizens~\cite{france_surveillance, netherlands_surveillance, kazak_surveillance}.  In some cases governments work in cooperation with ISPs; last year, reports gave evidence that AT\&T was in a surveillance partnership with the NSA~\cite{ATT_NSA}.  More recently, the Investigatory Powers Bill (IP Bill) in the UK, if passed, will require ISPs to store citizens' browsing history for a year, and allow intelligence agencies to collect bulk data on their citizens~\cite{uk_bill}.

Currently, governements are challenged by Internet routing and are taking efforts to control where their citizens' traffic travels.  Russia has issued multiple statements about securing their domestic Internet traffic after a routing error was realized, which showed that domestic Russian traffic was being routed through a China Telecom router in Frankfurt, Germany, and then through Stockholm, Sweden before returning to Russia~\cite{russia_secure_internet, routing_errors}.  Governments and users have been motivated more than ever since the Snowden revelations to avoid countries known for surveillance practices, specifically the United States.  In 2013, Deutsche Telekom AG (Germany's largest phone company) asked the government for help setting up a framework to keep German traffic within the country's borders in order to avoid other governments from spying on their Internet traffic~\cite{dte}.  More recently, the Safe Harbour agreement, an agreement that allows the free flow of data between the US and the EU, was struck down because it would give the NSA access to EU citizens' personal data~\cite{safe_harbour_illegal}.  At the moment, there is still no data transfer pact between the US and EU to allow data to cross the Atlantic, and both governing bodies are heavily divided on many crucial details~\cite{safe_harbour_undecided}.

In response to the Snowden revelations, Brazil has taken great measures to avoid Internet traffic from being wiretapped by the NSA.  A bill for Brazilian Internet civil rights was finalized in 2010, but it wasn't until after the NSA's mass surveillance was revealed that the bill was passed~\cite{brazil_history}.  Since then, the Brazilian President has made it clear that she wants to protect Brazilians' data privacy~\cite{brazil_break_from_US, brazil_conference, brazil_conference2, brazil_human_rights}.  Some of the actions that have been taken to avoid NSA surveillance include: building a 3,500 mile long fiber-optic cable from Fortaleza to Portugal (with no use of American vendors), pressing companies such as Google, Facebook, and Twitter (among others) to store data locally, switching it's dominant email system (Microsoft Outlook) to a state-developed system called Expresso~\cite{brazil_cable, brazil_us_companies}.  In other efforts, Brazil has been building Internet eXchange Points, which help grow Internet connectivity and performance~\cite{brazil_IXP1, brazil_IXP2}; it also allows Brazil to increase their connectivity with many other countries.  Brazil now has the largest national ecosystem of public Internet eXchange points in the world \annie{need to get this citation - it's at PAM 2016}~\cite{brazil_largest_IXP}.  It has also been shown that over the past few years, the number of ASes in Brazil that are connected internationally have grown significantly~\cite{brazil_international_ases}.  The case of Brazil shows that countries, governments, and Internet users are motivated to avoid surveillance conducted by other countries.

In this paper, we first shed light on how much traffic is currently transiting countries known for surveillance, with a focus on the US.  Despite the measures taken by different countries to avoid the US, we still see Internet traffic that transits the US.  We conduct a small measurement study that helps quantify the existing possibilities for state-sponsored surveillance.  We take Brazil as a case study, and analyze the country-level paths from machines in Brazil to the Brazil Alexa Top 100 domains.  Knowing that Brazil has taken action to avoid traffic transitting the US, wanted to find how much traffic is solely transitting the US, and how much traffic is destined for the US.  Using RIPE Atlas probes and the Digital Envoy geolocation service, we find \annie{once we can map the IPs to countries, we can put results here}~\cite{ripe_atlas, digital_envoy}.  

Next, we present a new system design for surveillance circumvention for Internet users, which finally gives the users some control over where their traffic is flowing.  Because popular web companies are opening or expanding their datacenters in Europe, there are more possible paths to get data~\cite{eu_datacenters}.  It is possible that a path from a client in Brazil to a datacenter in Europe does not pass through any country known for surveillance, but may be a longer path than the best path to a datacenter in the US. \annie{possibly add a figure to demonstrate this here?} Our system leverages a geographically diverse set of machines that act as proxies to access data from servers located in different geographic regions.  A client will first query an authoritative server for the best proxy to use in order to avoid a given country.  Then the authoritative server will respond with the IP address of the best proxy, which the client will then send it's request to.  The proxy will fetch the data from the domain's closest server and return the data to the client.  While this system is fairly simple, it cannot be replaced with existing systems such as Tor or VPN Gate; neither of these systems give the user control over country avoidance.  They also have different goals: Tor was built for anonymity and VPN Gate for anonymity and accessing restricted sites~\cite{tor, vpngate}.  The goal of our system is not anonymity, but country avoidance for surveillance circumvention.

This paper is organized as follows.  In the next section, we discuss how and where we collected our data.  We point out the advantages and disadvantages to existing datasets, and justify our decision.  In Section \ref{measure}, we design and execute a measurement study on the country-level paths of Brazil's Internet.  We describe our methodology, as well as results that show which countries Brazil's Internet traffic is traversing.  Next, Section \ref{architecture} introduces methods that Internet users, ISPs, and Internet services can use to circumvent surveillance.  We present a new system that allows users to control which countries their traffic transits (limitation to this system are also discussed).  In Section \ref{evaluation}, we evaluate our system and proposed methods for how well they avoid any given country.  Next, we discuss how our system differs from others and uniquely suits the purpose of country avoidance in Section \ref{discussion}, we review related work in Section \ref{related}, and conclude in Section \ref{conclusion}.
