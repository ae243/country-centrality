\section{Related Work}
\label{related}

\paragraph{Nation-state routing analysis.}  Recently, Shah and
Papadopoulos measured international BGP detours (paths that originate in
one country, cross international borders, and then return to the
original country)~\cite{shah2015characterizing}.  Using BGP routing
tables, they found 2 million detours in each month of their study (out
of 7 billion total paths), and they then characterized the detours based
on detour dynamics and persistence.  Our work differs by actively
measuring traceroutes (actual paths), as opposed to analyzing BGP
routes.  Obar and Clement analyzed traceroutes
that started and ended in Canada, but tromboned through the United
States, and argued that
this is a violation of Canadian network
sovereignty~\cite{obar2012internet}. 
Karlin et al. developed a framework for country-level
routing analysis to study how much influence each country has over
interdomain routing~\cite{karlin2009nation}.  This work measures the
centrality of a country using BGP routes and AS-path inference; in contrast, our work uses active 
measurements and measures avoidability of a given country. 

\paragraph{Mapping national Internet topologies.}  Roberts et
al. described a method for mapping national networks of ASes,
identifying ASes that act as points of control~\cite{roberts2011mapping}.  
%JEN: not clear we care about the "complexity" (whatever that means)
%, and measuring the complexity of the national network
Also, several studies have measured and classified the network within
a country, including
Germany~\cite{wahlisch2010framework,wahlisch2012exposing} and
China~\cite{zhou2007chinese}, or a country's interconnectivity within
a region or with the rest of the
world~\cite{bischof2015and,gupta2014peering,fanou2015diversity}.

\paragraph{Circumvention and Routing Systems.}  There has been research into
circumvention systems, particularly for censorship circumvention, which is 
complementary to our work, but not sufficient for surveillance circumvention.  
Existing circumvention systems generally rely on encryption, which does not 
prevent surveillance; prior research has shown that websites can be 
fingerprinted based on size, content, and location of third party resources, which 
reveals information about the content a user is accessing \cite{what_isps_can_see}.  Additionally, 
ISPs often execute a man-in-the-middle attacks on TLS connections to perform 
network management functions~\cite{mitm_isp}.  Therefore, while encryption can be 
used for censorship circumvention, additional measures must be taken for 
surveillance circumvention.  Some existing circumvention tools are Tor and VPNGate.
Tor is an anonymity system that uses three relays and layered encryption
to allow users to communicate anonymously~\cite{dingledine2004tor}.
VPNGate is a public VPN relay system aimed at circumventing national
firewalls~\cite{nobori2014vpn}.  Unfortunately, VPNGate does not allow a
client to choose any available VPN, which makes surveillance avoidance
harder.  Another system, Alibi Routing, is a peer-to-peer system that 
uses round-trip times to prove that that a client's packets did 
not traverse a forbidden country or region~\cite{levin2015alibi}; our work differs by measuring 
which countries a client's packets would (and does) traverse.  Our work then 
uses active measurements to determine the best path for a client wishing 
to connect to a server.  RON, Resilient Overlay Network, is an overlay network that 
routes around failures, whereas our overlay network routes around countries~\cite{andersen2001resilient}.

