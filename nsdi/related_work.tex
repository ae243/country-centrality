\section{Related Work}
\label{related}

\paragraph{Nation-state routing analysis.}  Recently, Shah and
Papadopoulos measured international BGP detours (paths that originate in
one country, cross international borders, and then return to the
original country)~\cite{shah2015characterizing}.  Using BGP routing
tables, they found 2 million detours in each month of their study (out
of 7 billion total paths), and they then characterized the detours based
on detour dynamics and persistence.  Our work differs by actively
measuring traceroutes (actual paths), as opposed to analyzing BGP
routes.  This difference is fundamental as BGP provides the AS path
announced in BGP update messages, which is not necessarily the same as
the actual path of data packets.  Obar and Clement analyzed traceroutes
that started and ended in Canada, but ``boomeranged'' through the United
States (``boomerang'' is another term for tromboning), and argued that
this is a violation of Canadian network
sovereignty~\cite{obar2012internet}.  Most closely related to our work,
Karlin et al. developed a framework for country-level
routing analysis to study how much influence each country has over
interdomain routing~\cite{karlin2009nation}.  This work measures the
centrality of a country to routing and uses AS-path inference to measure
and quantify country centrality, whereas our work uses active
measurements and measures avoidability of a given country. 

\paragraph{Mapping national Internet topologies.}  In 2011, Roberts et al. described
a method for mapping national networks of ASes, identifying ASes that
act as points of control in the national network, and measuring the
complexity of the national network~\cite{roberts2011mapping}.  There
have also been a number of studies that measured and classified the
network within a country.  Wahlisch et al. measured and classified the
ASes on the German Internet~\cite{wahlisch2010framework,
  wahlisch2012exposing}, Zhou et al. measured the complete
Chinese Internet topology at the AS level~\cite{zhou2007chinese}, and
Bischof et al. characterized the current state of Cuba's
connectivity with the rest of the world~\cite{bischof2015and}.
Interconnectivity has also been studied at the continent level; Gupta
et al. first looked at ISP interconnectivity within
Africa~\cite{gupta2014peering}, and it was studied later by Fanou et al.~\cite{fanou2015diversity}.

\paragraph{Circumvention and Routing Systems.}  There has been research into
circumvention systems, particularly for censorship circumvention, that
is related this work, but not sufficient for surveillance circumvention.
Tor is an anonymity system that uses three relays and layered encryption
to allow users to communicate anonymously~\cite{dingledine2004tor}.
VPNGate is a public VPN relay system aimed at circumventing national
firewalls~\cite{nobori2014vpn}.  Unfortunately, VPNGate does not allow a
client to choose any available VPN, which makes surveillance avoidance
harder.  Another system, Alibi Routing, is a peer-to-peer system that 
uses round trip times to prove that that a client's packets did 
not traverse a forbidden country or region~\cite{levin2015alibi}; our work differs by measuring 
which countries a client's packets would (and do) traverse.  Our work then 
uses active measurements to determine the best path for a client wishing 
to connect to a server.  RON, Resilient Overlay Networks, is an overlay network that 
routes around failures, whereas our overlay network routes around countries~\cite{andersen2001resilient}.

