\section{Design Goals}
\label{goals}
In light of the results presented in the previous section, we see that an 
overlay network can significantly help a client avoid any given country.  This motivates
 the design and implementation of our system, \system.  Here we highlight the main goals of 
\system{}, as well as challenges that are out of the scope of this work.

{\bf Foreign Country Avoidance.}  The primary goal of \system{} is to avoid a given
 country when accessing web content.  \system{} should provide clients a way to 
route around a specified country, while still being able to access the desired 
domain.  This calls for the role of measurement in the system design.  Systematizing 
the measurement methods discussed earlier in this paper is crucial to enabling the 
system to avoid any specified country.

{\bf Usability.} \system{} should be designed in a way that is accessible to and 
easy to use by clients around the world.  It should require as little effort by 
the client as possible.  This means that clients shouldn't have to download or 
install any software, they shouldn't have to run any measurements, and they 
should not have to understand how the system works.  This calls for a technique 
that will allow clients to multiplex between relays (proxies) based on which domain they 
are accessing, while doing so in a seamless, automated way.  The Proxy Autoconfiguration (PAC) 
file supports the functionality that is required by \system{} --- we discuss the details 
of PAC files in the next two sections.  

{\bf Scalability.}  This country avoidance system should be able to scale to 
large numbers of users.  Therefore, \system{} should be able to handle the addition
 of relays, as well as be cost-effective in terms of resources required. This requires 
clever measurement vantage points, such that each vantage point is representative of 
more than one client.  It is also aided by the PAC file, as this file allows the system to 
grow with increasing numbers of clients, and supports incremental deployment.

{\bf Non-goals.}  There are some challenges that \system{} does not attempt to 
solve.  The system does not address the notion of anonymity; it routes around 
countries (for reasons such as avoiding mass surveillance), but it does not 
attempt to keep users anonymous.  

Additionally, \system{} does not solve the problem of domestic surveillance (for 
example, a client in the United States attempting to avoid surveillance by the 
United States).  This is a challenging problem, and asks for future research, 
but this is out of the scope of our work.
