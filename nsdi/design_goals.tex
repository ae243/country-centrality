\section{Design Goals for \system{}}
\label{goals}

Our measurement results motivate 
 the design and implementation of a relay-based avoidance system,
\system{}, with the following design goals.

\paragraph{Country Avoidance.}  The primary goal of \system{} is to
avoid a given country when accessing web content.  \system{} should
provide clients a way to route around a specified country, when
accessing a domain.  This calls for the role of measurement in the
system design, and systematizing the measurement methods discussed
earlier in the paper.

\paragraph{Usability.} \system{} should require as little effort as
possible from clients.  Clients should not have to download
or install software, collect any measurements, or understand how the
system works.  This calls for a way for clients to automatically and
seamlessly multiplex between relays (proxies) based on different
destinations.  The Proxy Autoconfiguration (PAC) file supports this
function.

\paragraph{Scalability.}  This country avoidance system should be able to scale to 
large numbers of users.  Therefore, \system{} should be able to handle the addition
 of relays, as well as be cost-effective in terms of resources required. This requires 
clever measurement vantage points, such that each vantage point is representative of 
more than one client.  The PAC file allows the system to 
grow with the number of clients and also supports incremental deployment.

\paragraph{Non-goals.}  There are some challenges that \system{} does not attempt to 
solve.  The system does not address the notion of anonymity; it routes around 
countries (for reasons such as avoiding mass surveillance), but it does not 
attempt to keep users anonymous.  
\system{}, of course, does not address domestic surveillance (for 
example, a client in the U.S. attempting to avoid surveillance by the 
U.S.).

