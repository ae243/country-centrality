\section{Conclusion}
\label{conclusion}

We have measured Internet paths to characterize routing
detours that take Internet paths through countries that perform
surveillance.  Our findings show that paths commonly traverse known
surveillance states, even when they originate and end in a
non-surveillance state.  As a possible step towards a remedy, we have
investigated how clients can use overlay network relays to prevent routing detours through
unfavorable jurisdictions.  This method gives clients the power to
avoid certain countries, as well as help keep local traffic local.
Although some countries are completely avoidable, we find that some of
the more prominent surveillance states are the least avoidable.

We make country avoidance accessible to Internet users by designing 
and implementing \system{}, which employs overlay network relays to 
route Internet traffic around a given country.  Our evaluation shows 
that \system{} is successful at avoiding countries while performing 
as well, if not better, than taking default routes.

Our work presents several opportunities for follow-up studies and
future work. First, Internet paths continually
evolve; we will repeat this analysis over time and publish the results
and data on a public website, to help deepen our collective
understanding about how the evolution of Internet connectivity affects
transnational routes. Second, our analysis should be extended to study
the extent to which citizens in one country can avoid groups of
countries or even entire regions. Finally, although our results provide strong 
evidence for the existence of various transnational data flows, factors
such as uncertain IP geolocation make it difficult to provide clients
guarantees about country-level avoidance; developing techniques and
systems that offer clients stronger guarantees
is a ripe opportunity for future work.
