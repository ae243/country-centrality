\begin{abstract}
  As the Internet has grown up, it has also developed borders.  Certain
  countries now engage in interference, degradation, blocking, or
  surveillance of Internet traffic.  In response, individuals,
  organizations, and even entire countries are taking steps to control
  the geographic regions that their traffic traverses. For example, some
  countries are building local Internet Exchange Points (IXPs) to
  prevent traffic that originates and terminates in the country from
  taking detours through other foreign countries. Unfortunately, our
  measurements reveal that many such ongoing efforts are futile, for two
  reasons: local content is often hosted in foreign countries, and
  networks within a country often fail to peer with one another. Yet,
  our work offers hope: we also find that re-routing traffic through
  strategically placed overlay network relays can reduce transnational
  routing detours from \xxx{XX--YY\%}; using DNS open resolvers to help
  clients discover and use different replicas of the same service can
  also reduce transnational detours by \xxx{XX--YY\%}. Based on these
  findings, we design and implement \\system{}, a lightweight system that
  automatically routes a client's web traffic around specified countries
  with no modifications to client software (and in many cases with
  little performance overhead). Anyone can use \\system{} today; we have
  deployed long-running \\system{} Web proxies around the world, released the
  source code, and provided detailed instructions for configuring a client
 to use the system.
\end{abstract}
