\begin{abstract}
An increasing number of countries are passing laws that facilitate the
mass surveillance of Internet traffic. In response, governments and
citizens are increasingly paying attention to the countries that their
Internet traffic traverses. In some cases, countries are taking extreme
steps, such as building new Internet Exchange Points (IXPs), which allow networks to interconnect 
directly, and encouraging local interconnection
to keep local traffic local. We find that although many of these efforts
are extensive, they are often futile, due to the inherent lack of
hosting and route diversity for many popular sites. By measuring the country-level paths to 
popular domains, we characterize transnational routing detours.  We find that traffic is traversing known surveillance 
states, even when the traffic originates and ends in a country that does not conduct mass surveillance.  Then, we investigate how
clients can use overlay network relays and the open DNS resolver
infrastructure to prevent their traffic from traversing certain jurisdictions. We find
 that 84\% of paths originating in Brazil traverse the United States, 
but when relays are used for country avoidance, only 37\% of Brazilian paths 
traverse the United States.  Using the open DNS resolver infrastructure allows Kenyan clients 
to avoid the United States on 17\% more paths.  We then design and implement a system, RANSOM, that 
allows a client to route their traffic around a country they wish to avoid. Our evaluation 
shows that RANSOM achieves more country avoidance than default paths while having a
negligible impact on performance.
\end{abstract}
