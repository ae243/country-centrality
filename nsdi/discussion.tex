\section{Discussion}
\label{discussion}

\paragraph{Avoiding multiple countries.} 
We have studied only the extent to which Internet paths can be
engineered to avoid a {single} country.  Yet, avoiding a single country
may force an Internet path into {\em other} unfavorable
jurisdictions. This possibility suggests that we should also be
exploring the feasibility of avoiding multiple surveillance states (\eg,
the ``Five Eyes'') or perhaps even entire regions. 
%It is already clear
%that avoiding certain combinations of countries is not possible, at
%least given the current set of relays; for
%example, to avoid the US, Kenyan clients rely on the relay located in
%Ireland, so avoiding both countries is often impossible.

\paragraph{The evolution of routing detours and avoidance over time.}
Our study is based on a snapshot of Internet paths. Over time, paths
change, hosting locations change, IXPs are built, submarine cables are
laid, and surveillance states change.  Future work can and should
involve exploring how these paths evolve over time, and analyzing the
relative effectiveness of different strategies for controlling traffic flows.

%\paragraph{Isolating DNS diversity vs. path diversity.}
%In our experiments, the overlay network relays perform DNS lookups from
%geographically diverse locations, which provides some level of DNS
%diversity in addition to the path diversity that the relays inherently
%provide. This approach somewhat conflates the benefits of DNS diversity
%with the benefits of path diversity and in practice may increase
%clients' vulnerability to surveillance, since each relay is performing
%DNS lookups on each client's behalf. We plan to conduct additional
%experiments where the client relies on its local DNS resolver to map
%domains to IP addresses, as opposed to relying on the relays for both
%DNS resolution and routing diversity.

\paragraph{Additional \system{} Features.}  Additional features can be 
implemented at the relay to help preserve client privacy.  An example would 
be to use the relay as a mix, or to send out fake traffic to confuse an 
attacker that may be trying to perform traffic analysis at the relay.

The oracle could add additional 
steps in the decision chain introduced in Section \ref{multiplex} that take 
into account relay and path loads.  For example, if multiple relays provide 
a path to a domain that does not traverse the specified country, then the 
decision between the usable proxies could be determined based on current relay 
load or performance. 

Our current implementation of \system{} 
re-computes all paths once per hour.  This could be optimized to only re-compute 
paths when necessary.  For example, a BGP monitoring system could be implemented 
that alerts the oracle to a path change that affects any path currently in 
the system.  This could decrease the cost and computation of the system.
