\section{Implementation}
\label{implementation}

Our implementation uses VPN tunnels between the client and each relay.

VPN:
    Pros: Runs below the application level, so it's protocol-independent.  Provides encryption between client->relay, so ISP/government can't interfere.  Allows us clients to use arbitrary protocols without modifying an existing protocol stack.
    Cons: Heavier.  Could be more expensive (if we use a paid service).  Reuqires users to update iptables, build new tunnels.  Scalability (how many tunnels can be run in parallel?).  How many clients can a single openvpn server handle?

SOCKS Proxy:
    Pros: supports any kind of internet traffic
    Cons: different software/applications have to configured differently

HTTP Proxy:
    Pros: Inexpensive/free, simple
    Cons: Only for web content (HTTP or HTTPS), requires users to update configuration file for domain->proxy mappings.  How many clients cana single proxy server handle?


Discuss what each component needs, what tools we used, and why.  Justify VPN, AWS.

Discuss how the client caches domain->relay mappings in config.  Once a day we pull from oracle and update whole config file.

Discuss how we map traceroutes to country-level paths in the oracle and the server.

Include paragraph on implementation improvements (DNS Server instead of oracle, scalability, load-balancing, performance optimizations (when no relay is needed), granularity, etc.)
