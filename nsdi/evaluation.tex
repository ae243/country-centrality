\section{Evaluation}
Using the \system{} implementation, we evaluate the system on it's ability to avoid a given country, performance, and scalability in terms of storage and costs.

\subsection{Country Avoidance}
As the primary goal of the system is to provide country avoidance for a given 
country, we measured how much avoidance the system achieves.  We did so by first 
calculating the number of {\it default} paths that avoid a given country.  Then 
we added a single relay, and calculated how many domains the client could 
access without traversing through the given country.  This was repeated for 
the remaining two relays.  The evaluation was conducted under the condition that 
the client wished to avoid different countries when accessing the Netherlands top 
100 domains, and the results are shown in Figure \ref{fig:avoidance_eval}.  Each 
line represents the fraction of domains accessible while avoiding the country that 
the line represents.  For example, 46\% of domains are accessible without traversing 
the United States when \system{} is not being used (0 relays), and if \system{} is 
used, then 63\% of domains are accessible with traversing the United States.

\begin{figure}[b!]
\tiny
\centering
\includegraphics[width=.5\textwidth,height=6cm]{avoidance_n_relays}
\caption{How much avoidance different numbers and locations of relays achieve when a 
client located in the Netherlands wishes to avoid different countries.  Up to nine 
relays were used, but there was no additional avoidance provided by the 4 remaining 
relays (not included in this graph).}
\label{fig:avoidance_eval}
\end{figure}

It is evident that \system{} helps a client avoid a foreign country, as the 
fraction of domains accessible without traversing 
the specified country without \system{} is lower than with \system{}.  Additionally, 
it is clear that adding the first relay provides the greatest increase in 
provided avoidance, while subsequent relays provide a significantly 
smaller amount (or no) additional avoidance.

Figure \ref{fig:avoidance_eval} also clearly shows how much more difficult (or 
impossible) it is to avoid the United States than it is to avoid any other 
country.  Only 63\% of domains can be accessed while avoiding the United States, 
whereas almost all domains can be accessed while avoiding any other given 
country.  This confirms the results presented in Section \ref{avoid_results}, and 
emphasizes how crucial the systematization of the measurements is for enabling 
\system{}.

\subsection{Performance}
A system is not usable if the performance is significantly worse than what a user
is accustomed to.  To measure the performance of \system{}, we measure both 
the throughput and latency.

To measure throughput, we ran {\tt wget} for each 
of the top 100 domains from the client machine in the Netherlands, while 
using an oracle-generated PAC file.  Because multiple different relays could have been 
used to avoid a single domain, the oracle selected a random relay from those 
that would allow the client to avoid the country.  The oracle generated 
10 PAC files randomly picking a relay for domains that could have used 
different relays, and {\tt wget} was used for the top 100 domains for each 
PAC file generated.  Based on the {\tt wget} output, we calculate the number 
of seconds to access content using our system and take the average across the 
10 experiments. Figure \ref{fig:throughput} shows 
the CDF of the ratio of direct throughput to \system{} throughput. 

\begin{figure}[t]
\centering
\includegraphics[width=.5\textwidth]{throughput}
\caption{CDF of the ratio of RAN throughput to direct throughput.  
There were six outliers that had a ratio greater than 10.}
\label{fig:throughput}
\end{figure}

We can see that the throughput of \system{} is not significantly worse than that 
of default paths.  In fact, in some cases the performance of \system{} is {\it 
better} than that of default paths.  This could be a result of the relays 
keeping local traffic local, or due to a closer content replica being selected. 
These results show that \system{}'s performance is comparable to the performance 
of accessing domains without \system{}.

To measure the latency of \system{}, we ran a {\tt curl} command to each of the 
top 100 domains from the client machine in the Netherlands, while using the 10
oracle-generated PAC files. This provided the time to first byte (TTFB); we 
found the average TTFB for when accessing content using \system{} and 
found the TTFB when using direct paths; the results are shown in Figure \ref{fig:latency}.  

\begin{figure}[t]
\centering
\includegraphics[width=.5\textwidth]{latency}
\caption{CDF of Time to First Byte for both \system{} and for direct paths.}
\label{fig:latency}
\end{figure}

The median TTFB for direct paths is .0685, and for \system{} paths, it is .10075.  Additionally, 
the 90th percentile for direct and \system{} paths is .2253 and .40352, respectively.  This 
shows that the system's latency is greater by a factor of 2 (or 20-40ms).  \annie{Say if this is 
significant or not, especially as it relates to page load times.}

\subsection{Storage}
As the number of clients increase, and subsequently the number of paths being 
computed increases, the amount of storage must remain reasonable.  The storage 
used by paths can be calculated:

\[Storage(D,R,C) = (D x R) + 2(C x R) + (C x D) \]

D is the number of domains; R is the number of relays; C is representative of the number of 
clients.  While C {\it represents} clients, it is not the number of {\it actual} clients using the 
system --- it is the number of vantage points the system uses to measure paths 
from client locations.  For the prototype with a single client, the storage space for all 
paths computed is 480KB.  As there is a single PAC file for all clients in 
a country, C will grow much slower than if there was a different PAC file for 
each individual client.  There are 196 countries in the world today, and if 
paths and a PAC file were generated for each country, with 100 domains, and 
three relays, the storage would only be 94MB.  This provides plenty of storage 
for increasing the number of domains included in the PAC file or increasing 
the number of relays in the system.

\subsection{Costs}
In addition to storage, the cost of the measurements used in the system must 
be taken into account.  RIPE Atlas credits are a limited resource, and therefore 
we must earn more credits than we are spending on measurements.  The cost 
in credits follows the equation:

\[Credit\_Cost(D,R,C) = COST_{traceroute}((C x R) + (C x D))\]

Currently, the $COST_{traceroute}$ is 60, resulting in a prototype cost of 6,180 
credits, but because these paths are updated each hour, then 
the daily credit cost is 148,320 credits.  In return for hosting a RIPE Atlas 
probe, we earn 216,000 credits per day, which will support our existing 
prototype.  In order to provide for more clients, more domains, or more 
resources, we can tune the system to re-compute paths less frequently (only when necessary).
