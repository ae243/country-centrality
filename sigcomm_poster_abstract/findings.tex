\section{Preliminary Findings}

\paragraph{Hosting diversity.}
We start by looking at hosting diversity, more specifically, how many
countries a domain is hosted in.  More diversity should provide for the
potential to avoid more countries.  About half of the
Alexa Top 100 domains in the five countries studied are hosted in more
than one country. 

\paragraph{Routing diversity.}
Despite strong efforts made by some countries,
their traffic still traverses surveillance states, and is
subject to surveillance.  Over 50\% of the Alexa Top 100 domains in
Brazil and India are hosted in the United States, and over 50\% of the
paths from the Netherlands to the Alexa Top 100 domains transits the
United States.  About half of Kenyan traffic traverses the United States
and Great Britain.   

\paragraph{On the feasiblity of avoidance.}
By measuring which domains are accessible without traversing a given
country using open resolvers and an overlay network, we see that
there are ways to circumvent unfavorable countries.  Without these
country avoidance techniques, Brazilian traffic transitted Spain, Italy,
France, Great Britain, Argentina, Ireland (among others), but using the
overlay network, Brazilian clients could completely avoid these
countries for the top 100 domains.  The overlay network can be used to
keep local traffic local; by using relays in the client's country, less
traffic trombones out of the client's country.  The percentage of paths
from the United States to the top 100 domains decreases from 11.2\% to 1.3\% when
relays are used.   

\paragraph{Cause for concern.}
Unfortunately, some of the more prominent surveillance states are also
some of the {\textit least avoidable} countries.  Most countries are
highly dependent on the United States, a known surveillance state, and
not dependent on other countries.  Neither Brazil, India, Kenya, or the
Netherlands can completely avoid the United States with the country
avoidance techniques.  With the overlay network, both Brazilian and
Netherlands paths avoid the United States about 65\% of the time, and
the United States is completely unavoidable for about 10\% of the paths
because it is the only country where the content is hosted.  Kenyan traffic can
only avoid the United States on about 40\% of the paths from Kenya to
the top 100 domains.  On the other hand, the United States can avoid
every other country except for France and the Netherlands, and even then
they are avoidable for 99\% of the top 100 domains.   

% Our measurements and analysis show that Internet traffic is currently
% traversing surveillance states even when it originates a country that
% does not conduct surveillance.  After using open resolvers and an
% overlay network to measure country avoidance, we found that countries
% (and surveillance states) can be avoided, and local traffic can be kept
% local.   

