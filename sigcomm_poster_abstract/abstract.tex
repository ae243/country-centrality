% This is "sig-alternate.tex" V2.1 April 2013
% This file should be compiled with V2.5 of "sig-alternate.cls" May 2012
%
% This example file demonstrates the use of the 'sig-alternate.cls'
% V2.5 LaTeX2e document class file. It is for those submitting
% articles to ACM Conference Proceedings WHO DO NOT WISH TO
% STRICTLY ADHERE TO THE SIGS (PUBS-BOARD-ENDORSED) STYLE.
% The 'sig-alternate.cls' file will produce a similar-looking,
% albeit, 'tighter' paper resulting in, invariably, fewer pages.
%
% ----------------------------------------------------------------------------------------------------------------
% This .tex file (and associated .cls V2.5) produces:
%       1) The Permission Statement
%       2) The Conference (location) Info information
%       3) The Copyright Line with ACM data
%       4) NO page numbers
%
% as against the acm_proc_article-sp.cls file which
% DOES NOT produce 1) thru' 3) above.
%
% Using 'sig-alternate.cls' you have control, however, from within
% the source .tex file, over both the CopyrightYear
% (defaulted to 200X) and the ACM Copyright Data
% (defaulted to X-XXXXX-XX-X/XX/XX).
% e.g.
% \CopyrightYear{2007} will cause 2007 to appear in the copyright line.
% \crdata{0-12345-67-8/90/12} will cause 0-12345-67-8/90/12 to appear in the copyright line.
%
% ---------------------------------------------------------------------------------------------------------------
% This .tex source is an example which *does* use
% the .bib file (from which the .bbl file % is produced).
% REMEMBER HOWEVER: After having produced the .bbl file,
% and prior to final submission, you *NEED* to 'insert'
% your .bbl file into your source .tex file so as to provide
% ONE 'self-contained' source file.
%
% ================= IF YOU HAVE QUESTIONS =======================
% Questions regarding the SIGS styles, SIGS policies and
% procedures, Conferences etc. should be sent to
% Adrienne Griscti (griscti@acm.org)
%
% Technical questions _only_ to
% Gerald Murray (murray@hq.acm.org)
% ===============================================================
%
% For tracking purposes - this is V2.0 - May 2012

\documentclass{sig-alternate-05-2015}

\usepackage{url}
\def\UrlBreaks{\do\/\do-}
\usepackage{breakurl}
\usepackage[breaklinks]{hyperref}

\begin{document}

% Copyright
\setcopyright{acmcopyright}
%\setcopyright{acmlicensed}
%\setcopyright{rightsretained}
%\setcopyright{usgov}
%\setcopyright{usgovmixed}
%\setcopyright{cagov}
%\setcopyright{cagovmixed}


% DOI
%\doi{10.475/123_4}

% ISBN
%\isbn{123-4567-24-567/08/06}

%Conference
%\conferenceinfo{PLDI '13}{June 16--19, 2013, Seattle, WA, USA}

%\acmPrice{\$15.00}

%
% --- Author Metadata here ---
%\conferenceinfo{WOODSTOCK}{'97 El Paso, Texas USA}
%\CopyrightYear{2007} % Allows default copyright year (20XX) to be over-ridden - IF NEED BE.
%\crdata{0-12345-67-8/90/01}  % Allows default copyright data (0-89791-88-6/97/05) to be over-ridden - IF NEED BE.
% --- End of Author Metadata ---

\title{Nation-State Routing}
%
% You need the command \numberofauthors to handle the 'placement
% and alignment' of the authors beneath the title.
%
% For aesthetic reasons, we recommend 'three authors at a time'
% i.e. three 'name/affiliation blocks' be placed beneath the title.
%
% NOTE: You are NOT restricted in how many 'rows' of
% "name/affiliations" may appear. We just ask that you restrict
% the number of 'columns' to three.
%
% Because of the available 'opening page real-estate'
% we ask you to refrain from putting more than six authors
% (two rows with three columns) beneath the article title.
% More than six makes the first-page appear very cluttered indeed.
%
% Use the \alignauthor commands to handle the names
% and affiliations for an 'aesthetic maximum' of six authors.
% Add names, affiliations, addresses for
% the seventh etc. author(s) as the argument for the
% \additionalauthors command.
% These 'additional authors' will be output/set for you
% without further effort on your part as the last section in
% the body of your article BEFORE References or any Appendices.

\numberofauthors{4} %  in this sample file, there are a *total*
% of EIGHT authors. SIX appear on the 'first-page' (for formatting
% reasons) and the remaining two appear in the \additionalauthors section.
%
\author{
% You can go ahead and credit any number of authors here,
% e.g. one 'row of three' or two rows (consisting of one row of three
% and a second row of one, two or three).
%
% The command \alignauthor (no curly braces needed) should
% precede each author name, affiliation/snail-mail address and
% e-mail address. Additionally, tag each line of
% affiliation/address with \affaddr, and tag the
% e-mail address with \email.
%
% 1st. author
\alignauthor
Anne Edmundson\\
       \affaddr{Princeton University}\\
       \email{annee@cs.princeton.edu}
% 2nd. author
\alignauthor
Roya Ensafi\\
       \affaddr{Princeton University}\\
       \email{rensafi@cs.princeton.edu}
\and % use '\and' if you need 'another row' of author names
% 3rd. author
\alignauthor Jennifer Rexford\\
       \affaddr{Princeton University}\\
       \email{jrex@cs.princeton.edu}
% 4th. author
\alignauthor Nick Feamster\\
       \affaddr{Princeton University}\\
       \email{feamster@cs.princeton.edu}
}
% There's nothing stopping you putting the seventh, eighth, etc.
% author on the opening page (as the 'third row') but we ask,
% for aesthetic reasons that you place these 'additional authors'
% in the \additional authors block, viz.

\date{}
% Just remember to make sure that the TOTAL number of authors
% is the number that will appear on the first page PLUS the
% number that will appear in the \additionalauthors section.

\maketitle
\begin{abstract}
As an increasing number of countries pass laws that facilitate mass
surveillance of their citizens~\cite{france_surveillance,
  netherlands_surveillance, kazak_surveillance, uk_bill}.  Governments
and users have been motivated more than ever since the Snowden
revelations to avoid countries known for surveillance practices,
specifically the United States~\cite{russia_secure_internet,
  routing_errors, dte}.  More recently, the Safe Harbour agreement, an
agreement that allows the free flow of data between the US and the EU,
was struck down because it would give the NSA access to EU citizens'
personal data~\cite{safe_harbour_illegal, safe_harbour_undecided}.  When
Internet traffic enters a country, it becomes subject to those countries
laws.  As a result, users have more need than ever to determine---and
control--- which countries their traffic is traversing.

Certain countries such as Brazil have already taken impressive measures
to ensure that Internet traffic circumvents the United
States~\cite{brazil_history, brazil_break_from_US, brazil_conference,
  brazil_conference2, brazil_human_rights} including building a 3,500
mile long fiber-optic cable from Fortaleza to Portugal (with no use of
American vendors), pressing companies such as Google, Facebook, and
Twitter (among others) to store data locally, and switching its dominant
email system (Microsoft Outlook) to a state-developed system called
Expresso~\cite{brazil_cable, brazil_us_companies}.  Brazil is also
building Internet Exchange Points (IXPs)~\cite{brazil_IXP1}, now has the
largest national ecosystem of public Internet eXchange points in the
world~\cite{brazil_ixp_ecosystem}, and the number of internationally
connected ASes continues to grow~\cite{brazil_international_ases}.  

Unfortunately, these mechanisms alone may not be sufficient to cause
traffic to circumvent a particular country. The design of the Internet and routing protocols have no notion of national borders, and thus Internet traffic paths are determined without any regard to international crossings.  The BGP (Border Gateway Protocol) decides interdomain routing paths based on shortest and most preferred paths - not on which countries it will traverse.  This allows traffic to pass through countries that conduct surveillance even when the traffic originates (and possibly terminates) in a country that does not lawfully allow surveillance.  

Determining where a client's Internet traffic flows is complicated by the complexity of websites~\cite{butkiewicz2011understanding}.  Many websites also fetch content from other domains, which are most likely hosted in different locations.  Therefore, the client has to make additional web requests, which take different paths.  One initial web request can result in content being fetched from many servers located around the world, and to see where this traffic flows requires knowledge of all paths from the client to requested sources (and all the requested sources to the client).  As the number of domains, and therefore the number of paths, increase, the more possibilities for surveillance are introduced.

There are ways for clients to force their traffic to avoid potential surveillance by avoiding unfavorable jurisdictions.  One way is to use open resolvers; a client could query a foreign resolver, such that it appears as if the client is located in a geographically different location.  This different location could potentially be closer to a georeplicated server in a foreign country, and thus the path from the client to the server has the potential to avoid certain countries.  The same method could be used with resolvers that support EDNS; the client could spoof the client subnet with a foreign subnet, and therefore access a replica in a favorable country.

Unfortunately, this approach is unsuccessful if the service uses anycast IP addresses; this technique has increased in the past few years~\cite{cicalese2015characterizing}.  Even though the DNS lookup is in a different location than the client, the client will still be accessing data from the server that is closest, which provides the possibility of unwanted countries on the path between the client and the server.  

An alternative way for a client to keep their traffic from transitting a surveillance state is by using an overlay network.  With the use of relays in diverse geographic locations, a client could potentially avoid two ways that traffic could be surveilled by a foreign country.  The first way is if the unfavorable country is on the path from the client to the destination, and by using a relay in a different country, the traffic never passes through the unfavorable country.  The second way is if the closest content replica is in an unfavorable country, but by using a relay, the closest content replica to the relay is in a favorable country, and therefore the client accesses his content from a replica that is not in the unfavorable country.  Additionally, sometimes a client's traffic is domestic (ends in his own country), but traverses international borders.  Using relays in the client's country could potentially help keep domestic traffic local and prevent foreign countries from surveilling this traffic.

We start by looking at hosting diversity, more specifically, how many countries a domain is hosted in.  More diversity should provide for the potential to avoid more countries.  We found that about half of the Alexa Top 100 domains in the five countries studied are hosted in more than one country.

Given the diversity in hosting sites, there is reason to believe that clients can force their traffic to avoid certain countries using either open resolvers or an overlay network.  Before measuring how avoidable a country is, we measure the current state of routing traffic across borders.  Our study is focused on analyzing the traffic originating in five different countries: Brazil, Netherlands, Kenya, India, and the United States.  Using RIPE Atlas probes and the MaxMind geolocation service, we measure the country-level traffic paths for the Alexa Top 100 domains in each respective country.  Using the current state of routing as a baseline for comparison, we then measure how avoidable a given country is to a client in either Brazil, Netherlands, India, Kenya, or the United States, using open resolvers and using the overlay network.  Our contributions include:

\begin{itemize}
\item First (to our knowledge) in-depth measurement study of nation-state routing for Brazil, Netherlands, Kenya, India, and the United States.
\item Metrics for quantifying country avoidability.
\item Evaluation of the use of open resolvers and an overlay network as tools for country avoidance.
\end{itemize}

Our findings show that despite strong efforts made by some countries, their traffic still traverses surveillance states, and therefore is subject to surveillance.  Over 50\% of the Alexa Top 100 domains in Brazil and India are hosted in the United States, and over 50\% of the paths from the Netherlands to the Alexa Top 100 domains transits the United States.  About half of Kenyan traffic traverses the United States and Great Britain.  

By measuring which domains are accessible without traversing a specified country using open resolvers and an overlay network, we showed that there are ways to circumvent unfavorable countries.  Without these country avoidance techniques, Brazilian traffic transitted Spain, Italy, France, Great Britain, Argentina, Ireland (among others), but using the overlay network, Brazilian clients could completely avoid these countries for the top 100 domains.  The overlay network can be used to keep local traffic local; by using relays in the client's country, less traffic trombones out of the client's country.  The percentage of paths from Brazil to the top 100 domains decreases from 13.2\% to 9.7\% when relays are used.  Similarly, United States traffic trombones only 1.3\% with an overlay network, as compared to the 11.2\% without it.  

Unfortunately, some of the more prominent surveillance states are also some of the {\textit least avoidable} countries.  Most countries are highly dependent on the United States, a known surveillance state, and not dependent on other countries.  Neither Brazil, India, Kenya, or the Netherlands can completely avoid the United States with the country avoidance techniques.  With the overlay network, both Brazilian and Netherlands paths avoid the United States about 65\% of the time, and the United States is completely unavoidable for about 10\% of the paths because it is the only country where it is hosted.  Kenyan traffic can only avoid the United States on about 40\% of the paths from Kenya to the top 100 domains.  On the other hand, the United States can avoid every other country except for France and the Netherlands, and even then they are avoidable for 99\% of the top 100 domains.  

Our measurements and analysis show that Internet traffic is currently traversing surveillance states even when it originates a country that does not conduct surveillance.  After using open resolvers and an overlay network to measure country avoidance, we found that countries (and surveillance states) can be avoided, and local traffic can be kept local.  

\end{abstract}


%
% The code below should be generated by the tool at
% http://dl.acm.org/ccs.cfm
% Please copy and paste the code instead of the example below. 
%


%
% End generated code
%

%
%  Use this command to print the description
%
%\printccsdesc

% We no longer use \terms command
%\terms{Theory}

%\keywords{ACM proceedings; \LaTeX; text tagging}


%\end{document}  % This is where a 'short' article might terminate

%ACKNOWLEDGMENTS are optional

%
% The following two commands are all you need in the
% initial runs of your .tex file to
% produce the bibliography for the citations in your paper.
\bibliographystyle{abbrv}
\bibliography{sigproc}  % sigproc.bib is the name of the Bibliography in this case
% You must have a proper ".bib" file
%  and remember to run:
% latex bibtex latex latex
% to resolve all references
%
% ACM needs 'a single self-contained file'!
%
%APPENDICES are optional
%\balancecolumns
%\balancecolumns % GM June 2007
% That's all folks!
\end{document}
