% This is "sig-alternate.tex" V2.1 April 2013
% This file should be compiled with V2.5 of "sig-alternate.cls" May 2012
%
% This example file demonstrates the use of the 'sig-alternate.cls'
% V2.5 LaTeX2e document class file. It is for those submitting
% articles to ACM Conference Proceedings WHO DO NOT WISH TO
% STRICTLY ADHERE TO THE SIGS (PUBS-BOARD-ENDORSED) STYLE.
% The 'sig-alternate.cls' file will produce a similar-looking,
% albeit, 'tighter' paper resulting in, invariably, fewer pages.
%
% ----------------------------------------------------------------------------------------------------------------
% This .tex file (and associated .cls V2.5) produces:
%       1) The Permission Statement
%       2) The Conference (location) Info information
%       3) The Copyright Line with ACM data
%       4) NO page numbers
%
% as against the acm_proc_article-sp.cls file which
% DOES NOT produce 1) thru' 3) above.
%
% Using 'sig-alternate.cls' you have control, however, from within
% the source .tex file, over both the CopyrightYear
% (defaulted to 200X) and the ACM Copyright Data
% (defaulted to X-XXXXX-XX-X/XX/XX).
% e.g.
% \CopyrightYear{2007} will cause 2007 to appear in the copyright line.
% \crdata{0-12345-67-8/90/12} will cause 0-12345-67-8/90/12 to appear in the copyright line.
%
% ---------------------------------------------------------------------------------------------------------------
% This .tex source is an example which *does* use
% the .bib file (from which the .bbl file % is produced).
% REMEMBER HOWEVER: After having produced the .bbl file,
% and prior to final submission, you *NEED* to 'insert'
% your .bbl file into your source .tex file so as to provide
% ONE 'self-contained' source file.
%
% ================= IF YOU HAVE QUESTIONS =======================
% Questions regarding the SIGS styles, SIGS policies and
% procedures, Conferences etc. should be sent to
% Adrienne Griscti (griscti@acm.org)
%
% Technical questions _only_ to
% Gerald Murray (murray@hq.acm.org)
% ===============================================================
%
% For tracking purposes - this is V2.0 - May 2012

\documentclass{sig-alternate-05-2015}

\usepackage{url}
\def\UrlBreaks{\do\/\do-}
\usepackage{breakurl}
\usepackage[breaklinks]{hyperref}

\begin{document}

% Copyright
\setcopyright{acmcopyright}
%\setcopyright{acmlicensed}
%\setcopyright{rightsretained}
%\setcopyright{usgov}
%\setcopyright{usgovmixed}
%\setcopyright{cagov}
%\setcopyright{cagovmixed}


% DOI
%\doi{10.475/123_4}

% ISBN
%\isbn{123-4567-24-567/08/06}

%Conference
%\conferenceinfo{PLDI '13}{June 16--19, 2013, Seattle, WA, USA}

%\acmPrice{\$15.00}

%
% --- Author Metadata here ---
%\conferenceinfo{WOODSTOCK}{'97 El Paso, Texas USA}
%\CopyrightYear{2007} % Allows default copyright year (20XX) to be over-ridden - IF NEED BE.
%\crdata{0-12345-67-8/90/01}  % Allows default copyright data (0-89791-88-6/97/05) to be over-ridden - IF NEED BE.
% --- End of Author Metadata ---

\title{}
%
% You need the command \numberofauthors to handle the 'placement
% and alignment' of the authors beneath the title.
%
% For aesthetic reasons, we recommend 'three authors at a time'
% i.e. three 'name/affiliation blocks' be placed beneath the title.
%
% NOTE: You are NOT restricted in how many 'rows' of
% "name/affiliations" may appear. We just ask that you restrict
% the number of 'columns' to three.
%
% Because of the available 'opening page real-estate'
% we ask you to refrain from putting more than six authors
% (two rows with three columns) beneath the article title.
% More than six makes the first-page appear very cluttered indeed.
%
% Use the \alignauthor commands to handle the names
% and affiliations for an 'aesthetic maximum' of six authors.
% Add names, affiliations, addresses for
% the seventh etc. author(s) as the argument for the
% \additionalauthors command.
% These 'additional authors' will be output/set for you
% without further effort on your part as the last section in
% the body of your article BEFORE References or any Appendices.

\numberofauthors{4} %  in this sample file, there are a *total*
% of EIGHT authors. SIX appear on the 'first-page' (for formatting
% reasons) and the remaining two appear in the \additionalauthors section.
%
\author{
% You can go ahead and credit any number of authors here,
% e.g. one 'row of three' or two rows (consisting of one row of three
% and a second row of one, two or three).
%
% The command \alignauthor (no curly braces needed) should
% precede each author name, affiliation/snail-mail address and
% e-mail address. Additionally, tag each line of
% affiliation/address with \affaddr, and tag the
% e-mail address with \email.
%
% 1st. author
\alignauthor
Anne Edmundson\\
       \affaddr{Princeton University}\\
       \email{annee@cs.princeton.edu}
% 2nd. author
\alignauthor
Roya Ensafi\\
       \affaddr{Princeton University}\\
       \email{ensafi@cs.princeton.edu}
\and % use '\and' if you need 'another row' of author names
% 3rd. author
\alignauthor Jennifer Rexford\\
       \affaddr{Princeton University}\\
       \email{jrex@cs.princeton.edu}
% 4th. author
\alignauthor Nick Feamster\\
       \affaddr{Princeton University}\\
       \email{feamster@cs.princeton.edu}
}
% There's nothing stopping you putting the seventh, eighth, etc.
% author on the opening page (as the 'third row') but we ask,
% for aesthetic reasons that you place these 'additional authors'
% in the \additional authors block, viz.

\date{}
% Just remember to make sure that the TOTAL number of authors
% is the number that will appear on the first page PLUS the
% number that will appear in the \additionalauthors section.

\maketitle
\begin{abstract}
As an increasing number of countries pass laws that facilitate mass
surveillance of their citizens~\cite{france_surveillance,
  netherlands_surveillance, kazak_surveillance, uk_bill}.  Governments
and users have been motivated more than ever since the Snowden
revelations to avoid countries known for surveillance practices,
specifically the United States~\cite{russia_secure_internet,
  routing_errors, dte}.  More recently, the Safe Harbour agreement, an
agreement that allows the free flow of data between the US and the EU,
was struck down because it would give the NSA access to EU citizens'
personal data~\cite{safe_harbour_illegal, safe_harbour_undecided}.  When
Internet traffic enters a country, it becomes subject to those countries
laws.  As a result, users have more need than ever to determine---and
control--- which countries their traffic is traversing.

Certain countries such as Brazil have already taken impressive measures
to ensure that Internet traffic circumvents the United
States~\cite{brazil_history, brazil_break_from_US, brazil_conference,
  brazil_conference2, brazil_human_rights} including building a 3,500
mile long fiber-optic cable from Fortaleza to Portugal (with no use of
American vendors), pressing companies such as Google, Facebook, and
Twitter (among others) to store data locally, and switching its dominant
email system (Microsoft Outlook) to a state-developed system called
Expresso~\cite{brazil_cable, brazil_us_companies}.  Brazil is also
building Internet Exchange Points (IXPs)~\cite{brazil_IXP1}, now has the
largest national ecosystem of public Internet eXchange points in the
world~\cite{brazil_ixp_ecosystem}, and the number of internationally
connected ASes continues to grow~\cite{brazil_international_ases}.  

Unfortunately, these mechanisms alone may not be sufficient to cause
traffic to circumvent a particular country. 
We first measure the extent to which traffic that does not
originate or terminate in the United States traverses the country. We
focus on Brazil as a case study and analyze the country-level paths from
machines in Brazil to the Brazil Alexa Top 100 domains. Using RIPE Atlas
probes and the Digital Envoy geolocation service, we measure 36,833
traffic paths originating in Brazil to the Brazilian Alexa Top 100
domains. Despite Brazil's efforts to avoid the United States, much of
the traffic to popular destiantions still appears to transit the United
States. The United States is the destination for 28,196 of these paths,
suggesting that in these cases routing alone cannot achieve country
avoidance: rather, more extensive caching is needed to avoid the United
States. Additionally, 2,699 paths transit the United States en route to
a destination that is not in the United States.  Fortunately, many web
services are acquiring a global footprint~\cite{eu_datacenters}, will
eventually make it possible to retrieve data without sending traffic
through a particular country. Yet, even when communicating with a
geo-replicated service, users will still need simple, lightweight
mechanisms to control how their traffic flows to these sources of
content, even if the destination itself is not in a particular country.
\end{abstract}


%
% The code below should be generated by the tool at
% http://dl.acm.org/ccs.cfm
% Please copy and paste the code instead of the example below. 
%


%
% End generated code
%

%
%  Use this command to print the description
%
%\printccsdesc

% We no longer use \terms command
%\terms{Theory}

%\keywords{ACM proceedings; \LaTeX; text tagging}


%\end{document}  % This is where a 'short' article might terminate

%ACKNOWLEDGMENTS are optional

%
% The following two commands are all you need in the
% initial runs of your .tex file to
% produce the bibliography for the citations in your paper.
\bibliographystyle{abbrv}
\bibliography{sigproc}  % sigproc.bib is the name of the Bibliography in this case
% You must have a proper ".bib" file
%  and remember to run:
% latex bibtex latex latex
% to resolve all references
%
% ACM needs 'a single self-contained file'!
%
%APPENDICES are optional
%\balancecolumns
%\balancecolumns % GM June 2007
% That's all folks!
\end{document}
