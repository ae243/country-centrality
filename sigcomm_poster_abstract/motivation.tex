\section{Introduction}

When Internet traffic enters a country, it becomes subject to those
countries laws.  As a result, users have more need than ever to
determine---and control---which countries their traffic is traversing.
An increasing number of countries pass laws that facilitate mass
surveillance of their citizens~\cite{france_surveillance,
  netherlands_surveillance, kazak_surveillance, uk_bill}, governments
and citizens are increasingly motivated to divert their Internet traffic
from countries that perform surveillance (notably, the United
States~\cite{russia_secure_internet, routing_errors, dte}).
% More recently, the Safe Harbor agreement,
%which allows the free flow of data between the US and the EU, 
%was struck down because it would give the NSA access to EU citizens'
%personal data~\cite{safe_harbour_illegal, safe_harbour_undecided}.  

Many countries---notably, Brazil---are taking impressive measures reduce
the likelihood that Internet traffic transits the United
States~\cite{brazil_history, brazil_break_from_US, brazil_conference,
  brazil_conference2, brazil_human_rights} including building a 3,500
mile long fiber-optic cable from Fortaleza to Portugal (with no use of
American vendors); pressing companies such as Google, Facebook, and
Twitter (among others) to store data locally; and switching its dominant
email system (Microsoft Outlook) to a state-developed system called
Expresso~\cite{brazil_cable, brazil_us_companies}.  Brazil is also
building Internet Exchange Points (IXPs)~\cite{brazil_IXP1}, now has the
largest national ecosystem of public Internet eXchange points (IXPs) in
the world~\cite{brazil_ixp_ecosystem}, and the number of internationally
connected ASes continues to
grow~\cite{brazil_international_ases}. Brazil is not alone: IXPs are
proliferating in eastern Europe, Africa, and other regions, in part out
of a desire to ``keep local traffic local''. Building IXPs alone, of
course, cannot guarantee that Internet traffic for some service does not
enter or transit a particular country: Internet protocols have no notion
of national borders, and interdomain paths depend in large part on
existing interconnection business relationships (or lack thereof).  In
this poster, we present evidence to suggest that existing Internet
hosting and interdomain paths still make it difficult to avoid certain
countries for many popular websites. 

Our study focuses on analyzing the traffic originating in
five different countries: Brazil, Netherlands, Kenya, India, and the
United States.  Using RIPE Atlas probes and the MaxMind geolocation
service, we measure the country-level traffic paths for the Alexa Top
100 domains in each respective country.  Using the current state of
routing as a baseline for comparison, we then measure how avoidable a
given country is to a client in either Brazil, Netherlands, India,
Kenya, or the United States, using open resolvers and using the overlay
network.  Our contributions include: 

\begin{itemize}
\item The first in-depth measurement study of
  nation-state routing for Brazil, Netherlands, Kenya, India, and the
  United States. 
\item A preliminary evaluation of how open DNS resolvers and overlay
  network relays can help citizens and governments discover and use
  network paths that avoid certain countries.
\end{itemize}
\noindent
We find that hosting for many popular websites lacks diversity; in many
cases, even websites that are popular {\em locally} are hosted outside
the country where citizens are trying to access them. For example, more
than 50\% of the Alexa Top 100 domains in Brazil are hosted in the
United States. Internet paths also lack geographic diversity: About half
of the paths originating in Kenya to the most popular Kenyan websites
traverse the United States or Great Britain. Much of this phenomenon is
due to ``tromboning'', whereby an Internet path starts and ends in a
country, yet transits an intermediate country; for example, about 13\%
of the paths that we explored from RIPE Atlas nodes in Brazil to the
Alexa Top 100 in Brazil trombone through the United States. Fortunately,
our preliminary results suggest that the use of overlay network relays
to intentionally introduce network detours, and the use of open DNS
resolvers to discover hosting diversity can reduce tromboning and
generally help users select paths that avoid certain countries.

\section{Method}

Our measurement study tackles two questions: (1)~Which countries do {\em
  default} Internet routing paths traverse?; (2)~Why types of methods
can we use to increase hosting and path diversity to help governments
and citizens better control transnational Internet paths.

Determining where a client's Internet traffic flows is complicated by
the complexity of websites~\cite{butkiewicz2011understanding}.  Many
websites also fetch content from other domains, which are most likely
hosted in different locations.  Therefore, the client has to make
additional web requests, which take different paths.  One initial web
request can result in content being fetched from many servers located
around the world, and to see where this traffic flows requires knowledge
of all paths from the client to requested sources (and all the requested
sources to the client).  

A client can use DNS open resolvers to discover georeplication of a site
or service that can facilitate the avoidance of specific countries. For
example, a client can query an open DNS resolver in a foreign country to
discover a different georeplicated instance of a service; this technique
(or the use of EDNS client subnet) can allow the client to discover
different replicas of the same serviec. The path to this newly
discovered replica may assist in avoiding particular countries
(particularly if the client is trying to avoid the country of the
original hosting replica!).  The increasing use of IP anycast can
sometimes make this technique
insufficient~\cite{cicalese2015characterizing}, however, if the client
receives the same IP address regardless of the apparent origin of the
DNS query.  Another approach is to use overlay network relays, which can
prevent the client from traversing an unfavorable country by introducing
a path that detours sufficiently from the default. Additionally, using
relays in the client's country can somtimes help keep local traffic
local, by exploiting local paths that BGP does not select by default.


Given the diversity in hosting sites, there is reason to believe that
clients can force their traffic to avoid certain countries using either
open resolvers or an overlay network.  Before measuring how avoidable a
country is, we measure the current state of routing traffic across
borders.  

