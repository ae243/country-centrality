\section{Related Work}
\label{related}

\begin{figure*}[t]
\centering
\includegraphics[width=.9\textwidth]{Current-Traffic_fig}
\caption{Measurement pipeline to study Internet paths from countries to
  popular domains.}
\label{fig:pipeline1}
\end{figure*}

{\bf Nation-state routing analysis.}  Shah and
Papadopoulos recently measured international routing detours---paths that originate
in
one country, cross international borders, and then return to the
original country---using public Border Gateway Protocol~(BGP) routing tables~\cite
{shah2015characterizing}. 
The study discovered 2~million detours each month out
of 7 billion paths.
%JEN: hard to understand.  doesn't seem important.
%the study also characterized the detours based
%on detour dynamics and persistence.  
Our work differs by {\em actively}
measuring Internet paths using traceroute, yielding a more precise (and accurate) measurement of the paths, %traffic is
%likely to take
as opposed to analyzing BGP
routes.  Obar and Clement analyzed traceroutes
that started and ended in Canada, but tromboned through the United
States~\cite{obar2012internet}. 
Karlin \ea{} developed a framework for country-level
routing analysis to study how much influence each country has over
interdomain routing~\cite{karlin2009nation}.  This work measures country
centrality using BGP routes and AS-path inference; in contrast, our work uses active 
measurements and measures avoidability of a given country. Several studies have also characterized network paths {\em
within} a country, including
Germany~\cite{wahlisch2010framework,wahlisch2012exposing} and
China~\cite{zhou2007chinese}, or a country's interconnectivity %within a region
%or with the rest of theworld~
\cite{bischof2015and,gupta2014peering,fanou2015diversity,roberts2011mapping}; these studies
focus on paths within a country, as opposed to paths that traverse multiple countries.

%{\bf Mapping national Internet topologies.}  %Roberts \ea{} developed a method
%for mapping national networks and identifying ASes that act as points of
%control~\cite{roberts2011mapping}.   %JEN: not clear we care about the
% Several studies have characterized network paths {\em
%within} a country, including
%Germany~\cite{wahlisch2010framework,wahlisch2012exposing} and
%China~\cite{zhou2007chinese}, or a country's interconnectivity %within a region
%or with the rest of theworld~
%\cite{bischof2015and,gupta2014peering,fanou2015diversity,roberts2011mapping}; these studies
%focus on paths within a country, as opposed to paths that traverse multiple countries.

{\bf Routing overlays and Internet architectures.} Alibi Routing uses
round-trip times to prove that that a client's packets did  not traverse a
forbidden country or region~\cite{levin2015alibi,levin_detour}; \system{} differs by
measuring  which countries a client's packets traverse.  Our
work uses active measurements to determine the best path for a client
wishing  to connect to a server, whereas Alibi Routing uses the speed of light and 
circle distance to calculate minimum RTTs for a packet to traverse a region.  Because Alibi Routing 
uses this technique, it is almost impossible for a client to provable avoid a neighboring region or country.  
Lastly, we see \system{} as a compliment to Alibi Routing; Alibi Routing may result in no paths {\it provably} avoiding 
a given region, but that does not mean there are no paths that avoid a given region.  \system{} is a good alternative 
to determine if there are any paths, based on active measurements, that do not traverse a given region.  Recently, 
researchers proposed DeTor, which applies Alibi Routing techniques 
to Tor~\cite{dingledine2004tor}, to prove that a Tor circuit does not traverse a forbidden region~\cite{levin_detour}. 
As DeTor uses Alibi Routing techniques, it suffers from the same limitations as these 
techniques; because DeTor is applied to the Tor network, it raises the chances of overloading 
certain relays that are in a position to be on many circuits for clients wishing to avoid certain 
countries.  Additionally, DeTor fails to consider 
reverse paths in the calculation of region avoidance, causing the results to state provable 
avoidance, when the forbidden region may actually be on the reverse path. RON, Resilient Overlay Network, is an
overlay network that  routes around failures~\cite{andersen2001resilient}, whereas our overlay network
routes around countries.  ARROW introduces a
model that allows users to route around ISPs~\cite{peter2015one}, but requires
ISP participation, making it considerably more difficult to deploy than
\system{}; additionally, ISPs may span multiple countries. %ARROW also aims to improve fault-tolerance, robustness, and
%security, rather than explicitly attempting to avoid certain countries; ARROW
%provides mechanisms to avoid individual ISPs, but such a mechanism is at a
%different level of granularity, because an ISP may span multiple countries.
Zhang \ea{} presented SCION, a ``clean-slate'' Internet architecture that
provides route control, but requires fundamental
changes to the Internet architecture. %failure isolation, and explicit trust information for
%communication~\cite{zhang2011scion}; SCION, however, requires fundamental
%changes to the Internet architecture, whereas \system{} is deployable today.

%{\bf Circumvention systems.}  Certain tools, such as anonymous
%communications systems or VPNs~\cite{dingledine2004tor,nobori2014vpn,piotrowska2017loopix,van2015vuvuzela,wolinsky2012dissent,tyagi2016stadium,corrigan2015riposte,kwon2016atom,kwon2016riffle,
%gelernter2016two}, use a combination of
%encryption and overlay routing to allow clients to avoid censorship and surveillance. Tor is
%an anonymity system that uses three relays and layered encryption to allow
%users to communicate anonymously~\cite{dingledine2004tor}.  In contrast,
%\system{} does not aim to achieve anonymity; instead, its aim is to ensure
%that traffic does not traverse a specific  country, a goal that Tor cannot
%achieve.  Even tools like Tor do not inherently thwart surveillance: Tor is
%vulnerable to traffic correlation attacks and some attacks are possible even
%on encrypted user traffic. Recently, researchers proposed DeTor, which applies Alibi Routing techniques 
%to Tor, to prove that a Tor circuit does not traverse a forbidden region~\cite{levin_detour}. 
%As DeTor uses Alibi Routing techniques, it suffers from the same limitations as these 
%techniques; because DeTor is applied to the Tor network, it raises the chances of overloading 
%certain relays that are in a position to be on many circuits for clients wishing to avoid certain 
%countries.  Additionally, DeTor fails to consider 
%reverse paths in the calculation of region avoidance, causing the results to state provable 
%avoidance, when the forbidden region may actually be on the reverse path.  %Another system, VPNGate, is a public VPN relay system aimed at
%circumventing national firewalls~\cite{nobori2014vpn}. Unfortunately, VPNGate
%does not allow a client to choose any available VPN endpoint, which makes it more
%difficult for a user to ensure that traffic avoids a particular country.  
%Neither of these systems explicitly avoid or route around countries. Additionally, existing circumvention systems generally rely on
%encryption, which is different from \system{}; prior research has shown that
%websites can be fingerprinted based on size, content, and location of third
%party resources, which  reveals information about the content a user is
%accessing \cite{what_isps_can_see}.  Finally, ISPs often execute man-in-the-middle attacks on TLS connections to perform network-management
%functions~\cite{mitm_isp}.

