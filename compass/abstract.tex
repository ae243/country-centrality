\begin{abstract}
While the growth of the Internet has fostered more efficient communications 
around the world, there is a large digital divide between 
Western countries and the rest of the world.  Countries such as Brazil, 
China, and Saudi Arabia have questioned and criticized America's Internet hegemony.  
This paper studies the extent to which various countries rely on the United States 
and other Western countries to connect to popular Internet destinations in those 
countries.  Unfortunately, our measurements reveal that underserved regions are 
dependent on North American and Western European regions for two reasons: local 
content is often hosted in foreign countries (such as the United States and the 
Netherlands), and networks within a country often fail to peer with one  another. 
Fortunately, we also find that routing traffic through strategically  placed relay
nodes can in some cases reduce the number of transnational routing detours by
more than a factor of two, which subsequently reduces the dependence of underserved 
regions on other regions. Based on these findings, we design and implement Region-Aware Networking, 
\system{}, a lightweight system that routes a client's web traffic around
specified countries with no modifications to client software (and in many
cases with little performance overhead). %Anyone can use \system{} today; we
%have deployed long-running \system{} relays around the world, released the
%source code, and provided instructions for clients to use the system.
\end{abstract}
