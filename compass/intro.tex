\section{Introduction}
\label{sec:intro}

As the Internet continues to grow, the increasing social, economic, and cultural 
hegemony of Western regions over the rest of the world have led to a digital divide.  This 
divide inhibits connectivity, transparency, and the equal exchange of ideas~\cite{hegemony}. In 
recent years, we have seen various countries and regions, such as Brazil, China, and Saudi Arabia, push back against the United States hegemony in the global Internet~\cite{hegemony2, hegemony_china,
hegemony_EU}.  

This paper specifically studies the routing differences between American/Western European regions and 
underserved regions by measuring and analyzing the international routing detours exhibited when 
accessing popular content. The consequences of these international routing detours include performance 
degradation, increased costs, surveillance, and censorship.  Previous work has shown that {\it tromboning}
paths---paths that start and end in the same country, but also traverse a
foreign country---are common~\cite{shah2015characterizing,gupta2014peering}, especially in 
underserved regions.  

In this paper, we study two questions: (1)~Which countries do {\em   default}
Internet routing paths traverse?; (2)~What methods can  help governments (or citizens, 
ISPs, etc.) better control transnational Internet paths and reduce dependence on the United States and 
Europe to transit Internet traffic?  We {\em actively
measure} the paths originating in five countries to the most popular
websites in each of these respective countries.
Our analysis focuses on five countries---Brazil,
Kenya, India, the Netherlands, and the United States. Brazil, Kenya, and India are representative of 
underserved regions in the world, and the Netherlands and the United States represent 
more dominant global powers.   

In contrast to previous work, we measure router-level
forwarding paths that traffic actually traverses, as opposed to analyzing Border
Gateway Protocol (BGP)
routes~\cite{karlin2009nation,shah2015characterizing}, which can provide at
best an indirect estimate of country-level paths to sites.
Although BGP routing can offer some information about paths, it does not
necessarily reflect the path that traffic actually takes, and it only provides
AS-level granularity, which is often too coarse to make strong statements
about which countries that traffic is traversing.  In contrast, we measure
routes from RIPE Atlas probes~\cite{ripe_atlas} in each country to the Alexa
Top~1000 domains for each country; we directly measure the paths not only to
the websites corresponding to themselves, but also to the sites hosting any
third-party content on each of these sites.

Although using direct measurements provides these benefits, there are a number of 
challenges associated with determining which countries a
client's traffic is traversing.  First,
performing direct measurements is more costly than passive analysis of BGP
routing tables; RIPE Atlas, in particular, limits the rate at which one can
perform measurements.  As a result, we had to be strategic about the origins
and destinations that we selected for our study. We study five
geographically diverse countries,  focusing on countries in each region that
are either underserved or more dominant in the global Internet.  Second, IP
geolocation---the process of determining the geographic location of an IP
address---is notoriously challenging, particularly for IP addresses that
represent Internet infrastructure, rather than end-hosts~\cite{gharaibeh2017look}. We address this
inaccuracy by making conservative estimates of the extent of routing detours,
and by recognizing that our goal is not to pinpoint a precise location for an
IP address as much as to achieve accurate reports of {\em significant} off-path 
detours to certain countries or regions. (Section~\ref{datasets} explains
our method in more detail; we also explicitly highlight ambiguities in our
results.) Finally, the asymmetry of Internet paths can also make it difficult
to analyze the countries that traffic traverses on the reverse path from
server to client; our study finds that country-level paths are often
asymmetric, and, as such, our findings represent a lower bound on
transnational routing detours.

We first {\em characterize the current state of transnational Internet
routing detours} (Section~\ref{datasets}).  We explore hosting diversity 
by first measuring the Alexa Top 1000 domains and comparing the location of 
path endpoints to that of the Alexa Top 100 domains; we find that there is no 
significant difference between the results in the two domain sets, and therefore 
focus on the Alexa Top 100 domains {\it and all third party domains}.  We find 
that only 45\% of the Alexa Top 100 domains in Brazil are hosted 
in more than one country (other countries studied showed similar results); in many cases,
that country is one that clients may want to avoid. Second, even if
hosting diversity can be improved, routing can still force traffic
through a small set of countries. Despite strong efforts made by some countries to ensure their
traffic does not transit certain countries~\cite{brazil_history,
  brazil_break_from_US, brazil_conference, brazil_conference2,
  brazil_human_rights}, their traffic still does so.  For example, over 50\% of the
top domains in Brazil and India are hosted in the United States, and
over 50\% of the paths from the Netherlands to the top domains transit
the United States.  About half of Kenyan paths to the top domains
traverse the United States and Great Britain (but the same half does not
traverse both countries).  Much of this phenomenon is due to
``tromboning'', whereby an Internet path starts and ends in the same country,
yet transits an intermediate country; for example, about 13\% of the
paths that we explored from Brazil tromboned through the United States.
Infrastructure alone is not enough. ISPs in respective regions
need better incentives to interconnect with one another to ensure
that local traffic stays local.

Second, we {explore the extent to which
clients can avoid relying on certain countries to popular destinations} by using overlay network relays to route 
Internet traffic around a given country (Section~\ref{avoid_results}).  Our results demonstrate that this technique can
be effective for clients in certain countries; of course, the effectiveness of
this approach naturally depends on where content is hosted for that country
and the diversity of Internet paths between ISPs in that country and the
respective hosting sites. For example, our results show that clients in Brazil
can completely avoid Spain, Italy, France, Great Britain, Argentina, and
Ireland (among others), even though the default paths to many popular
Brazilian sites traverse these countries. We also find that some of the most
independent regions are also some of the least avoidable regions.
For example, many countries depend on ISPs in the United States for connectivity to popular sites and content.
Additionally, overlay network relays can increase performance by keeping local
traffic local: by using relays in the country, fewer paths trombone
out of the client's country.

Finally, we {\em design, implement, and deploy Region-Aware Networking, \system{}, a system of overlay network relays that allows
a client to access web content while minimizing the her dependence on a specified
country} (Section~\ref{system_design}).  We implemented \system{} for end-users, 
but ISPs can also deploy \system{} proxies to gain more routing independence as a service 
to its customers.  Our evaluation shows that \system{} can effectively route around many different countries and
introduces minimal performance overhead.
