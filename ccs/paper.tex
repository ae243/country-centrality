% This is "sig-alternate.tex" V2.1 April 2013
% This file should be compiled with V2.5 of "sig-alternate.cls" May 2012
%
% This example file demonstrates the use of the 'sig-alternate.cls'
% V2.5 LaTeX2e document class file. It is for those submitting
% articles to ACM Conference Proceedings WHO DO NOT WISH TO
% STRICTLY ADHERE TO THE SIGS (PUBS-BOARD-ENDORSED) STYLE.
% The 'sig-alternate.cls' file will produce a similar-looking,
% albeit, 'tighter' paper resulting in, invariably, fewer pages.
%
% ----------------------------------------------------------------------------------------------------------------
% This .tex file (and associated .cls V2.5) produces:
%       1) The Permission Statement
%       2) The Conference (location) Info information
%       3) The Copyright Line with ACM data
%       4) NO page numbers
%
% as against the acm_proc_article-sp.cls file which
% DOES NOT produce 1) thru' 3) above.
%
% Using 'sig-alternate.cls' you have control, however, from within
% the source .tex file, over both the CopyrightYear
% (defaulted to 200X) and the ACM Copyright Data
% (defaulted to X-XXXXX-XX-X/XX/XX).
% e.g.
% \CopyrightYear{2007} will cause 2007 to appear in the copyright line.
% \crdata{0-12345-67-8/90/12} will cause 0-12345-67-8/90/12 to appear in the copyright line.
%
% ---------------------------------------------------------------------------------------------------------------
% This .tex source is an example which *does* use
% the .bib file (from which the .bbl file % is produced).
% REMEMBER HOWEVER: After having produced the .bbl file,
% and prior to final submission, you *NEED* to 'insert'
% your .bbl file into your source .tex file so as to provide
% ONE 'self-contained' source file.
%
% ================= IF YOU HAVE QUESTIONS =======================
% Questions regarding the SIGS styles, SIGS policies and
% procedures, Conferences etc. should be sent to
% Adrienne Griscti (griscti@acm.org)
%
% Technical questions _only_ to
% Gerald Murray (murray@hq.acm.org)
% ===============================================================
%
% For tracking purposes - this is V2.0 - May 2012

\documentclass{sig-alternate-05-2015}

\usepackage{color}
\usepackage{subcaption}
\captionsetup{compatibility=false}
\newcommand{\system}{{SENSOR}}
\graphicspath{{./figures/}}
\newtheorem{thm}{Goal}
\newcommand{\annie}[1]{\emph{\color{red}(#1)}}


\begin{document}

% Copyright
\setcopyright{acmcopyright}
%\setcopyright{acmlicensed}
%\setcopyright{rightsretained}
%\setcopyright{usgov}
%\setcopyright{usgovmixed}
%\setcopyright{cagov}
%\setcopyright{cagovmixed}


% DOI
%\doi{10.475/123_4}

% ISBN
%\isbn{123-4567-24-567/08/06}

%Conference
%\conferenceinfo{PLDI '13}{June 16--19, 2013, Seattle, WA, USA}

%\acmPrice{\$15.00}

%
% --- Author Metadata here ---
%\conferenceinfo{WOODSTOCK}{'97 El Paso, Texas USA}
%\CopyrightYear{2007} % Allows default copyright year (20XX) to be over-ridden - IF NEED BE.
%\crdata{0-12345-67-8/90/01}  % Allows default copyright data (0-89791-88-6/97/05) to be over-ridden - IF NEED BE.
% --- End of Author Metadata ---

\title{\annie{we need a new title}}
%
% You need the command \numberofauthors to handle the 'placement
% and alignment' of the authors beneath the title.
%
% For aesthetic reasons, we recommend 'three authors at a time'
% i.e. three 'name/affiliation blocks' be placed beneath the title.
%
% NOTE: You are NOT restricted in how many 'rows' of
% "name/affiliations" may appear. We just ask that you restrict
% the number of 'columns' to three.
%
% Because of the available 'opening page real-estate'
% we ask you to refrain from putting more than six authors
% (two rows with three columns) beneath the article title.
% More than six makes the first-page appear very cluttered indeed.
%
% Use the \alignauthor commands to handle the names
% and affiliations for an 'aesthetic maximum' of six authors.
% Add names, affiliations, addresses for
% the seventh etc. author(s) as the argument for the
% \additionalauthors command.
% These 'additional authors' will be output/set for you
% without further effort on your part as the last section in
% the body of your article BEFORE References or any Appendices.

\numberofauthors{4} %  in this sample file, there are a *total*
% of EIGHT authors. SIX appear on the 'first-page' (for formatting
% reasons) and the remaining two appear in the \additionalauthors section.
%
\author{
% You can go ahead and credit any number of authors here,
% e.g. one 'row of three' or two rows (consisting of one row of three
% and a second row of one, two or three).
%
% The command \alignauthor (no curly braces needed) should
% precede each author name, affiliation/snail-mail address and
% e-mail address. Additionally, tag each line of
% affiliation/address with \affaddr, and tag the
% e-mail address with \email.
%
% 1st. author
%\alignauthor
%Anne Edmundson\\
%       \affaddr{Princeton University}\\
%       \email{annee@cs.princeton.edu}
% 2nd. author
%\alignauthor
%Roya Ensafi\\
%       \affaddr{Princeton University}\\
%       \email{ensafi@cs.princeton.edu}
%\and % use '\and' if you need 'another row' of author names
% 3rd. author
%\alignauthor Jennifer Rexford\\
%       \affaddr{Princeton University}\\
%       \email{jrex@cs.princeton.edu}
% 4th. author
%\alignauthor Nick Feamster\\
%       \affaddr{Princeton University}\\
%       \email{feamster@cs.princeton.edu}
}

\date{}

\maketitle
\begin{abstract}
  We present \system{}, a system that allows Internet users to discover
  end-to-end Internet paths that avoid countries of their choosing. We
  conduct a large-scale measurement study to demonstrate that Internet
  paths often ``trombone'' to third-party countries, often to countries
  where laws may make users more vulnerable to surveillance than they
  would be in their home country. We then present the design and
  implementation of \system{}, an overlay network that allows users to
  quickly discover efficient end-to-end paths that avoid certain
  countries. \name{} operates with minimal modifications to client configuration and
  software and scales to a large number of users and web services.
\end{abstract}


%
% The code below should be generated by the tool at
% http://dl.acm.org/ccs.cfm
% Please copy and paste the code instead of the example below. 
%
\begin{CCSXML}
<ccs2012>
<concept>
<concept_id>10002978.10003014</concept_id>
<concept_desc>Security and privacy~Network security</concept_desc>
<concept_significance>300</concept_significance>
</concept>
</ccs2012>
\end{CCSXML}

\ccsdesc[300]{Security and privacy~Network security}
%
% End generated code
%

%
%  Use this command to print the description
%
\printccsdesc

% We no longer use \terms command
%\terms{Theory}

%\keywords{ACM proceedings; \LaTeX; text tagging}

\section{Introduction}
\label{intro}

There are no restrictions on which national jurisdictions BGP paths must or must not cross.  There is also no way for an Internet user to know or decide which countries her Internet traffic is transitting.  Internet routing is usually studied at the AS (Autonomous System) level - not the nation-state level.  An AS typically controls traffic as it transits it's internal network and then defines policies for filtering, monitoring, and routing traffic to the next AS.  It is becoming increasingly common for governments to require certain actions of ASes, such as enabling wiretapping or performing censorship.  Once Internet traffic enters a country, it is subject to that country's requirements, and subsequently the corresponding ASes' actions.  More and more countries are either trying or have already passed laws that allow for mass surveillance on their citizens~\cite{france_surveillance, netherlands_surveillance, kazak_surveillance}.  Recently, the Investigatory Powers Bill (IP Bill) in the UK, if passed, will require ISPs to store citizens' browsing history for a year, and allow intelligence agencies to collect bulk data on their citizens~\cite{uk_bill}.

Currently, governments are becoming more suspicious about where their citizens' traffic is going, and are facing the challenge of controlling which countries transit the traffic.  Governments and users have been motivated more than ever since the Snowden revelations to avoid countries known for surveillance practices, specifically the United States~\cite{russia_secure_internet, routing_errors, dte}.  More recently, the Safe Harbour agreement, an agreement that allows the free flow of data between the US and the EU, was struck down because it would give the NSA access to EU citizens' personal data~\cite{safe_harbour_illegal, safe_harbour_undecided}.

In response to the Snowden revelations, Brazil has taken great measures to avoid Internet traffic from being wiretapped by the NSA~\cite{brazil_history, brazil_break_from_US, brazil_conference, brazil_conference2, brazil_human_rights}.  Some of the actions that have been taken to avoid NSA surveillance include: building a 3,500 mile long fiber-optic cable from Fortaleza to Portugal (with no use of American vendors), pressing companies such as Google, Facebook, and Twitter (among others) to store data locally, switching its dominant email system (Microsoft Outlook) to a state-developed system called Expresso~\cite{brazil_cable, brazil_us_companies}.  In other efforts, Brazil has been building Internet eXchange Points, which help grow Internet connectivity and performance~\cite{brazil_IXP1, brazil_IXP2}; it also allows them to increase their connectivity with many other countries.  Brazil now has the largest national ecosystem of public Internet eXchange points in the world~\cite{brazil_ixp_ecosystem}.  It has also been shown that over the past few years, the number of ASes in Brazil that are connected internationally have grown significantly~\cite{brazil_international_ases}.  The case of Brazil shows that countries, governments, and Internet users are motivated to avoid surveillance conducted by other countries.

Determining which countries transit Internet traffic, which countries host the only server for a given domain, and which countries are avoidable for a given domain are challenging problems.  The first challeng arises from determining the country-level path that traffic takes; paths can be collected from either the data-plane - IP-level paths - or the control-plane - AS-level paths.  Our work uses the data-plane to analyze traffic paths; in order to determine which countries are on the path, the IP-level path must be mapped to a country-level path.  This is challenging because there is no ground truth for geolocation data; there is no guarantee of accuracy or completeness.  To address this challenge, we use different geolocation tools to increase our confidence in the country-level paths we collect.  More challenges arise in determining how to avoid countries; websites are complex and can fetch data from multiple third parties, which are likely located in different geographic locations, and therefore cross different countries' borders.  The increased usage of anycast IP complicates country avoidance because it gives clients less information about the location of the server from which they are accessing content.  

In this paper, we first shed light on how much traffic is currently transiting countries known for surveillance, with a focus on the US.  Despite the measures taken by different countries to avoid the US, we still see Internet traffic that transits the US.  We conduct a measurement study that helps quantify the existing possibilities for state-sponsored surveillance.  We take Brazil as a case study, and analyze the country-level paths from machines in Brazil to the Brazil Alexa Top 100 domains.  Knowing that Brazil has taken actions to avoid traffic transitting the US, we measure how much traffic is solely transitting the US, and how much traffic is destined for the US.  These are two distinct scenarios.  If the US is solely a country on the path (and not the end point), then there is the possibility for avoiding the US.  If the domain is solely hosted in the US, and therefore destined for the US, then the US cannot be avoided for the given domain.  Using RIPE Atlas probes and the Digital Envoy geolocation service, we find that out of 36,833 traffic paths originating in Brazil to the Brazilian Alexa Top 100 domains, 2,699 are solely transitting the US and the US is the destination for 28,196 of them (this does not necessarily mean that the US is the only possible destination, but it is the destination for Brazilian Internet users)~\cite{ripe_atlas, digital_envoy}.  

Next, we present and implement a new system for surveillance circumvention for Internet users, which finally gives the users some control over where their traffic is flowing.  Because popular web companies are opening or expanding their datacenters in Europe, there are more possible paths to get data~\cite{eu_datacenters}.  It is possible that a path from a client in Brazil to a datacenter in Europe does not pass through any country known for surveillance, but may be a longer path than the best path to a datacenter in the US, as shown in Figure~\ref{fig:intro}.  Our system leverages a geographically diverse set of relay machines that act as proxies to access data from servers located in different geographic regions.  A client will first query an oracle for the best relay to use in order to avoid a given country.  Then the oracle will respond with the IP address of the best relay, which the client will then send it's request to.  The relay will fetch the data from the domain's closest server and return the data to the client. \annie{Add a couple highlights of the results/evaluation section when we get there.}

\begin{figure}
\centering
\includegraphics[width=.5\textwidth]{intro_fig}
\caption{A shorter path to a server in a country known for surveillance (U.S.), and a longer path to a georeplicated server in Germany.  The longer path may be more preferred by the client because it doesn't traverse a country with known surveillance practices.}
\label{fig:intro}
\end{figure}

This paper is organized as follows.  In the next section, we describe our research goals, and the challenges in achieving them.  In Section~\ref{datasets} we discuss how and where we collected our data.  We point out the advantages and disadvantages of existing datasets, and justify our decision.  In Section \ref{measure}, we design and execute a measurement study on the country-level paths of Brazil's Internet.  We describe our methodology, as well as results that show which countries Brazil's Internet traffic is traversing.  Next, Section \ref{architecture} introduces SYSTEM, which allows Internet users, ISPs, and Internet services to avoid specified countries, and therefore circumvent surveillance.  Then we explain our implementation of SYSTEM in Section \ref{implementation}.  In Section \ref{evaluation}, we evaluate our system and proposed methods for how well they avoid any given country.  We discuss how our system differs from others and uniquely suits the purpose of country avoidance in Section \ref{discussion}, we review related work in Section \ref{related}, and conclude in Section \ref{conclusion}.

\section{Research Goals}
\label{problem}

The design of the Internet and routing protocols have no notion of national borders, and thus Internet traffic paths are determined without any regard to international crossings.  Issues arise when Internet traffic flows through countries which have different data privacy and surveillance laws; which geographic locations does one traffic flow pass through and which laws should govern it?  The BGP (Border Gateway Protocol) decides interdomain routing paths based on shortest and most preferred paths - not on which countries it will traverse.  This allows traffic to pass through countries that conduct surveillance even when the traffic originates (and possibly terminates) in a country that does not lawfully allow surveillance.  

Determining where a client's Internet traffic flows is complicated by the complexity of websites~\cite{butkiewicz2011understanding}.  Many websites also fetch content from other domains, which are most likely hosted in different locations.  Therefore, the client has to make additional web requests, which take different paths.  One initial web request can result in content being fetched from many servers located around the world, and to see where this traffic flows requires knowledge of all paths from the client to requested sources (and all the requested sources to the client).  Figure~\ref{fig:domains} shows the number of subsequent requests that are made from an initial web request for the Brazilian Alexa Top 100 domains.  As the number of domains, and therefore the number of paths, increase, the more possibilities for surveillance are introduced.

\begin{figure}
\centering
\includegraphics[width=.5\textwidth]{subsequent_request_hist}
\caption{A histogram of the number of third party requests are made by each initial domain request for the Brazilian Alexa Top 100 domains.}
\label{fig:domains}
\end{figure}

\begin{thm}
Measure how much Internet traffic traverses countries that are known for surveillance, and quantify how much possible surveillance can be conducted.  Measuring the country-level routing paths from clients to the Alexa Top 100 domains (and the subsequent domain requests); collect data on how much traffic is solely transmitted by countries that conduct surveillance and how much traffic has a destination in one of these countries.
\end{thm}

There are naive ways for clients to force their traffic to avoid potential surveillance.  One way is using EDNS (Extension mechanisms for DNS); a client could spoof his own client subnet in the DNS request, such that it appears as if the client is located in a geographically different location.  This different location could potentially be closer to a georeplicated server in a foreign country, and thus the path from the server to the client has a greater likelihood of avoiding certain countries.  The same method could be used with open resolvers in geographically different locations.  Figure~\ref{fig:resolvers} shows a scenario where this method would help clients avoid a country.  

\begin{thm}
Measure and quantify to what extent a country is avoidable to a client in a different country by spoofing the client subnet field in EDNS requests.
\end{thm}

Unfortunately, this approach is unsuccessful if the service uses anycast IP addresses; this technique has increased in the past few years~\cite{cicalese2015characterizing}.  IP anycast is the case where a set of servers share the same standard unicast IP address, despite being in geographically diverse locations.  Even though the DNS lookup is in a different location than the client, the client will still be accessing data from the server that is closest, which provides the possibility of unwanted countries on the path between the server and the client.  

%\begin{figure*}
%\centering
%\begin{subfigure}[t]{.5\textwidth}
%  \centering
%  \includegraphics[width=.4\linewidth]{no_open_resolver}
%  \caption{A possible path when a client uses a local resolver.}
%  \label{fig:local_resolver}
%\end{subfigure}%
%\begin{subfigure}[t]{.5\textwidth}
%  \centering
%  \includegraphics[width=.7\linewidth]{open_resolver}
%  \caption{A possible path when a client uses an open resolver in a geographically different location.}
%  \label{fig:open_resolver}
%\end{subfigure}%
%\caption{Different paths are shown when a client uses a local resolver (Figure~\ref{fig:local_resolver}) vs. a geographically distant open resolver (Figure~\ref{fig:open_resolver}.)}
%\label{fig:resolvers}
%\end{figure*}

\begin{thm}
Measure and quantify to what extent a country is avoidable to a client in a different country by using geographically diverse relays (an overlay network).
\end{thm}

\begin{thm}
Measure and quantify how much local traffic can be kept local (and not cross international borders) by using geographically diverse relays (an overlay network).
\end{thm}

\section{State of Surveillance}
\label{surv}
We focused our study on five different countries, and for each, we actively measured and analyzed traffic that originated there.  These five countries were chosen for specific reasons and we present them here.  We also discuss countries that currently conduct surveillance; this exploration is not exhaustive, but highlights countries that are passing new surveillance laws and countries that have strict surveillance practices already.   

\subsection{Studied Countries}
We selected Brazil, Netherlands, Kenya, India, and the United States for the following reasons.

\paragraph{Brazil.} It has been widely publicized that Brazil is actively trying to avoid having their traffic transit the United States.  They have been building IXPs, deploying underwater cables to Europe, and pressuring large U.S. companies to host content within Brazil~\cite{brazil_history, brazil_break_from_US, brazil_conference,
  brazil_conference2, brazil_human_rights, brazil_cable, brazil_us_companies, brazil_IXP1}.  This effort to avoid traffic transitting a specific country led us to investigate whether their efforts have been successful or not.

\paragraph{Netherlands.}  We selected to study the Netherlands for two reasons: 1) the Netherlands is beginning to emerge as a site where servers are located for cloud services, such as Akamai, and 2) the Netherlands is where a large IXP is located (AMS-IX). Analyzing the Netherlands will allow us to see what effect AMS-IX and the emergence of cloud service hosting has on their traffic.

\paragraph{Kenya.} Prior research on the interconnectivity of Africa~\cite{gupta2014peering, fanou2015diversity} led us to explore the characterization an African country's interconnectivity.  We chose Kenya for few reasons: 1) it has more than one IXP, 2) it has high Internet access and usage (for the East African region), and 3) it is a location with many submarine cable landing points~\cite{kenya_nigeria, teams}.

\paragraph{India.}  India has one of the highest number of Internet users in Asia, second only to China, which has already been well-studied~\cite{tsui2003panopticon, wang2010discourse}.  With such a high number of Internet users, and presumably a large amount of Internet traffic, we study India to see where this traffic is going.

\paragraph{United States.}  We chose to study the United States because of how inexpensive it is to host domains there, as well as the prevalence of Internet and technology companies located there.

\subsection{Surveillance States}

%Some countries have created agreements on data sharing, while others want to control their citizens' online presence.  
When analyzing which countries Internet traffic traverse, special attention should be given to countries that may be unfavorable because of their surveillance laws.  Some of the countries that are currently conducting surveillance are the ``Five Eyes,'' ~\cite{lander2004international, eyeswideopen} the United States, Canada, United Kingdom, New Zealand, and Australia, as well as France, Germany, Poland, Hungary, Russia, Ukraine, Belarus, Kyrgyzstan, and Kazakhstan.  

\paragraph{Five Eyes.} The ``Five Eyes'' participants are the United States National Security Agency (NSA), the United Kingdom's Government Communications Headquarters (GCHQ), Canada's Communications Security Establishment Canada (CSEC), the Australian Signals Directorate (ASD), and New Zealand's Government Communications Security Bureau (GCSB)~\cite{eyeswideopen}.  According to the original agreement, the agencies can: 1) collect traffic; 2) acquire communications documents and equipment; 3) conduct traffic analysis; 4) conduct cryptanalysis; 5) decrypt and translate; 6) acquire information about communications organizations, procedures, practices, and equipment.  The agreement also implies that all five countries will share all intercepted material by default.  The agencies work so closely that the facilities are often jointly staffed by members of the different agencies, and it was reported ``that SIGINT customers in both capitals seldom know which country generated either the access or the product itself.''~\cite{lander2004international}.

\begin{table}[t!]
\centering
\caption{A sample of countries that are actively conducting surveillance at different levels of intensity.}
\label{surv_table}
\begin{tabular}{|p{1.4cm}|p{1.6cm}|p{1.6cm}|p{1.6cm}|}
\hline
Survillance stage & Collecting Metadata Phone \& Internet & Demanding ISPs to Participate & No Need for Court Order \\
\hline\hline
France            & \checkmark~\cite{francesurv, francesurv2} & \checkmark~\cite{francesurv} &   \\ \hline
Germany           & \checkmark~\cite{germansurv}   &             &                              \\ \hline
Poland            & \checkmark~\cite{francesurv2}      &         &  \checkmark~\cite{francesurv2}            \\ \hline
Hungry            & \checkmark~\cite{francesurv2}        &             & \checkmark~\cite{francesurv2} \\ \hline
Russia            & \checkmark~\cite{francesurv2}    & \checkmark~\cite{russiasurv, russiasurv2}    &  \\ \hline
%....              &      &        &   \\    \hline
\end{tabular}
\end{table}

A number of other countries are passing laws to facilitate mass surveillance.  These laws have differing levels of intensity, which can be seen in Table \ref{surv_table}; the countries with the least intense surveillance laws are listed at the top of the table, and those with the more intense laws are listed at the bottom.  These countries, along with the ``Five Eyes'' participants should be flagged when characterizing transnational detours in the following section.

%\paragraph{France.} France recently passed a new surveillance law that authorizes the government closely monitor the mobile phone and Internet communications of French citizens.  Additionally, the law requires ISPs to install ``black boxes'' that are designed to collect and analyze metadata on the Internet usage of millions of people.  The law allows surveillance without much oversight and the conditions under which the law's powers can be used are vague~\cite{francesurv}. The French Parliament has also made it easier for the government to access encrypted data for criminal investigations by enhancing penalties against companies that refuse to cooperate~\cite{francesurv2}.

%\paragraph{Germany.} A new data retention law was recently passed in Germany that will allow law enforcement agencies to access metadata of phone calls and Internet connections~\cite{germansurv}.  Prior to the bill, telecom providers only retained data for business purposes, but the bill will modify the German Telecommunications Act and require telecom providers to retain traffic data on phone calls and Internet connections.  The data collected includes IP addresses of users as well as the date and time of Internet connections.

%\paragraph{Poland.}  Poland now has stricter surveillance laws by ammending the Police Act.  This gives Polish authorities access to metadata without court approval~\cite{francesurv2}.

%\paragraph{Hungary.}  Hungary has pushed surveillance even further by proposing a new bill that makes it a crime for service providers to use encryption-based application or software.  It also forces ISPs to build back doors that give the government access to data.  Under the bill, ISPs are required to collect metadata on anyone who uses encryption, and it criminalizes the refusal of companies to give collected data to the government~\cite{francesurv2}.

%\paragraph{Russia.} Russia has had a surveillance infrastructure for years: SORM (System of Operative-Investigative Measures).  SORM involves a series of black boxes that ISPs must install; the providers are required to pay for the SORM equipment and installation and they are denied access to the box~\cite{russiasurv}.  If an ISP fails to cooperate, it is fined, and if the violation persists, then its license may be revoked.  Former Soviet Republics have followed in Russia's footsteps: Belarus, Ukraine, and Kyrgyzstan have adopted national interception systems modeled after SORM~\cite{russiasurv}.  Many popular sites, such as Facebook, Google, and Twitter are not hosted in Russia, which challenges surveillance.  In the past year, Russia has passed a data localization law.  This law states that any company that collects personal information from Russian users must stor their data on servers within the country - the main targets of the law were Facebook, Google, and Twitter~\cite{russiasurv2}. 

\section{Measurement Infrastructure}
\label{datasets}
For this study, we made careful choices about the measurement infrastructure.  These included which vantage points within a country to use, which servers to use as relays, and which method to use for geolocating IP addresses.  We also introduce our measurement methods for measuring where current Internet traffic is traveling to, how well open DNS resolvers help clients avoid specific countries, and how well relays help clients avoid specific countries.

\subsection{Resources}
\subsubsection{Vantage Points}
To our knowledge, the publicly available traceroute datasets suitable for our goal are from iPlane~\cite{madhyastha2006iplane} and CAIDA (Center for Applied Internet Data Analysis)~\cite{caida}.  The iPlane project uses PlanetLab ~\cite{planetlab} nodes to run traceroutes to a random set of IP addresses that cover all BGP atoms.  This project also has historical data as far back as 2006.  Unfortunately, because iPlane uses PlanetLab nodes, which have been shown to mostly use the Global Research and Education Network (GREN), the traceroutes run from PlanetLab nodes will not be representative of typical Internet users' traffic paths~\cite{banerjee2004interdomain}.  The other publicly available dataset, from CAIDA, ran traceroutes from different vantage points around the world, but to randomized destination IP addresses that cover all /24s.  This is also not sufficient for what we wanted to measure because a typical Internet user is going to access a domain that will be locally resolved; the user will not input a specific IP address in their browser.  Therefore, we chose to run active measurements that would be most representative of an Internet user.  We chose to run DNS and traceroute measurements from RIPE Atlas probes, which are hosted all around the world and in many different settings, include home networks~\cite{ripe_atlas}.  The first advantage of using RIPE Atlas is that the probes can use the local DNS resolver, which would give us the best estimate of the traceroute destination.  The next advantage is that we can control the parameters of traceroute; we specified to use paris traceroute as well as conducting both ICMP and TCP traceroutes.  We discuss more of our parameter selection and methodology in Section~\ref{measure}.

Our study looks at the Alexa Top 100 domains in different countries, as well as the 3rd party domains that are requested as part of an original web request.  To obtain these 3rd party domains we {\tt curl} each of the Top 100 domains, but we must do so from within the country we are studying.  There is no current functionality to {\tt curl} from RIPE Atlas probes, so we established a VPN connection within each of these countries to {\tt curl} each domain and extract the 3rd party domains.

\subsubsection{Servers}
As one of our research goals is to quantify how well the use of an overlay network will help clients avoid surveillance, we need access to a set of relays.  We used eight Amazon EC2 instances, one in each geographic region (United States, Ireland, Germany, Singapore, South Korea, Japan, Australia, Brazil), as well as 4 VPS machines (France, Spain, Brazil, Singapore).  The conjunction of these two sets of machines allow us to evaluate surveillance avoidance with a geographically diverse set of relays.

\subsubsection{Geolocation Data}
Previous work has shown that there are fundamental challenges in deducing a geographic location from an IP address, despite using different methods such as DNS names of the target, network delay measurements, and host-to-location mapping in conjunction with BGP prefix information~\cite{padmanabhan2001investigation}.  Because the focus of this work is on measuring and avoiding surveillance, and not on geolocation algorithms, we used a pre-existing geolocation service: MaxMind~\cite{maxmind}.

\subsection{Measurement Pipeline: Current Traffic}
\label{pipeline1}
Our methods for measuring where traffic paths go use the data-plane; we analyze the reported hops of traceroute measurements to find which countries are on the path from a client in Country X to a popular domain.  For this study, we used RIPE Atlas probes in Country X, specifically the set of probes that had unique ASes in the country.  The destinations for traceroutes are the Country X Alexa Top 100 domains, as well as the third party domains within the response bodies of the 100 domains.  There are three main steps to our measurement procedure: traceroute generation and collection, transformation of traceroutes to country-level paths, and path analysis.

{\bf Step 1.} The first step is to {\tt curl} each of the Country X Alexa Top 100 domains, and extract the third party domains from the response body.  These third party domains will each be fetched by the client when the original webpage is rendered in the browser.  Next, the RIPE Atlas probes in Country X will locally resolve each domain (and third party domain) by running a DNS measurement.  The local resolution is representative of a DNS response that an Internet user in Brazil would receive.  Once the DNS responses are received for all DNS measurements, the IP addresses are consolidated into a list of /24 subnets that cover the set of IP addresses; this is done because all IP addresses in a /24 network should geolocate to the same country.  The last part of this step is running traceroutes from the RIPE Atlas probes to a single IP address in each of the /24 subnets calculated from the DNS responses.  The measurements were run using paris traceroute and each (probe, destination IP) pair was used twice: once using ICMP traceroute and once using TCP traceroute.  

{\bf Step 2.}  Step 2 generates country-level paths from the set of traceroutes produced during Step 1.  Using MaxMind, each IP address was geolocated at a country granularity.  This resulted in a set of country-level paths.

{\bf Step 3.}  After calculating country-level paths, we perform different data analysis to find where traffic travels.  We look for countries that are on the path between the client and the destination.  These countries have the potential to conduct surveillance on the traffic of clients in Country X.  A different analysis can show the countries in which the path ends, representing where the content is hosted.  It's important to note that the destinations found in this study could change if the content is georeplicated in other countries.  On the otherhand, if it is not georeplicated, then the destination's country is unavoidable.  We study this more closely in Section \ref{metrics}.  Lastly, we look at how much tromboning occurs - a tromboning path is one that starts and ends in Country X, but also traverses at least one other country.  These paths are interesting because they should not be exiting the country if the destination is in the country of origin.  

\subsection{Measurement Pipeline: Country Avoidance with Open Resolvers}
\label{pipeline2}
If content is replicated in different parts of the world, open DNS resolvers located around the globe can potentially help clients circumvent surveillance states.  To measure this, we first pick a set of open resolvers to query, then query with Country X's Alexa Top 100 domains, and traceroute from a client in Country X to the DNS responses given by the open resolvers.  These three steps are explained in more detail below.

{\bf Step 1.}  Similar to the method in Section~\ref{pipeline1}, the first step is to {\tt curl} each of the Country X Alexa Top 100 domains, and extract the third party domains from the response body.  Next, a set of open DNS resolvers are selected that are geographically diverse; the set includes resolvers 10 different countries -- the same countries in which the relays (servers) are located (United States, Brazil, Ireland, France, Germany, Spain, Singapore, South Korea, Japan, Australia).  We query each open resolver for every domain and third party domain that was collected.

{\bf Step 2.}  After receiving the set of IP address from DNS responses, a VPN connection to Country X is established.  Through this connection, traceroute measurements are run to every IP address returned by the open resolvers. The result is a set of IP-level paths, which are then geolocated using MaxMind, and provides a set of country-level paths.

{\bf Step 3.}  After mapping the IP-level paths to country-level paths, we analyze the countries on each path.  We look for known surveillance states that are on the path from the client to the destination, where the content is hosted, and which countries are unavoidable.  We compare these results to those of the current state of nation-state routing to determine if open resolvers improve country avoidability.

\subsection{Measurement Pipeline: Country Avoidance with Relays}
Using an overlay network may help clients route around unfavorable countries or access content that is hosted in a more favorable country.  First, we select relays that are geographically diverse, run traceroute measurements from Country X to each relay and from each relay to the Country X Alexa Top 100 domains.  Based on these two sets of traceroutes, we can run analyses to determine if they improve country avoidability.

{\bf Step 1.}  First, we select a set of relays; there is at least one relay in each country that we used an open resolver in: United States, Brazil, Ireland, France, Germany, Spain, Singapore, South Korea, Japan, and Australia.  Next, a VPN connection to Country X is established, and traceroute measurements are run to each relay.

{\bf Step 2.}  The second step starts with {\tt curl}ing each of the Country X Alexa Top 100 domains, and extracting the third party domains from the response bodies.  After establishing an {\tt ssh} connection to each relay, we run traceroute measurements to each of the domains and third party domains.  At this point, we have two sets of traceroutes: 1) from Country X to all relays, and 2) from all relays to all domains.

{\bf Step 3.}  Like the previous measurement pipelines, the traceroutes are mapped from IP-level paths to country-level paths using MaxMind.

{\bf Step 4.}  The last step is analyzing the two sets for country avoidance.  We measure which countries are avoidable, how often they are avoidable, and how much better relays can achieve country avoidance in comparison to open resolvers.

\subsection{Avoidability Metrics}
\label{metrics}
We introduce a new metric and algorithm to measure how often a client in Country X can avoid a specified Country Y.  Using the proposed metric and algorithm, we can compare how well the different methods achieve country avoidance for a specified client Country X and unfavorable Country Y.

{\bf Avoidability Metric.}  This metric quantifies how often traffic can avoid Country Y when it originates in Country X.  Avoidability is explained as the fraction of paths that start in Country X and do not transit Country Y; more formally:

\[Avoidability(X,Y) = \frac{paths_{X,\bcancel{Y}}}{paths_{X}}\]

where $paths_{\bcancel{Y}}$ represent the paths from Country X that do not pass through Country Y, and $paths_{X}$ represent all paths that originate from Country X. The resulting value will be in the range [0,1], where 0 means the country is unavoidable for all of the domains in our study, and 1 means the client can avoid Country Y for all domains in our study.  This metric is used on country-level paths calculated during the measurements in Sections~\ref{pipeline1} and \ref{pipeline2}.  \annie{This may need to be updated to include domain weights based on the power law}

{\bf Avoidability Algorithm with Relays.}  Measuring the avoidability of a country Y from a client in Country X using relays has two components: 1) is Country Y on the path from the client in Country X to the relay?  2) is Country Y on the path from the relay to the domain?  For every domain, our algorithm checks if there exists at least one path from the client Country X through any relay and on to the domain, and does not transit Country Y.  The psuedo-code for the algorithm is shown in Algorithm~\ref{avoid_algo}.

\begin{algorithm}
\caption{My algorithm}\label{avoid_algo}
\begin{algorithmic}[1]
\Procedure{MyProcedure}{}
\State $\textit{stringlen} \gets \text{length of }\textit{string}$
\State $i \gets \textit{patlen}$
\BState \emph{top}:
\If {$i > \textit{stringlen}$} \Return false
\EndIf
\State $j \gets \textit{patlen}$
\BState \emph{loop}:
\If {$\textit{string}(i) = \textit{path}(j)$}
\State $j \gets j-1$.
\State $i \gets i-1$.
\State \textbf{goto} \emph{loop}.
\State \textbf{close};
\EndIf
\State $i \gets i+\max(\textit{delta}_1(\textit{string}(i)),\textit{delta}_2(j))$.
\State \textbf{goto} \emph{top}.
\EndProcedure
\end{algorithmic}
\end{algorithm}

The output of the algorithm is a value in the range [0,1] that can be compared to the output of the Avoidability Metric described above.  

{\bf Upperbound on Avoidability.}  While the Avoidability metric and algorithm provide a method to quantify how avoidable Country Y is from a client in Country X, it may be the case that a number of domains are hosted in Country Y, so the Avoidance value for these countries would never reach 1.0.  For this reason, we measured the upperbound on avoidance for given pair of (Country X, Country Y) that represents the best case value for avoidance.  The pseudocode for the algorithm is shown in Algorithm \ref{upperbound_algo}.

\begin{algorithm}
\caption{My algorithm}\label{upperbound_algo}
\begin{algorithmic}[1]
\Procedure{MyProcedure}{}
\State $\textit{stringlen} \gets \text{length of }\textit{string}$
\State $i \gets \textit{patlen}$
\BState \emph{top}:
\If {$i > \textit{stringlen}$} \Return false
\EndIf
\State $j \gets \textit{patlen}$
\BState \emph{loop}:
\If {$\textit{string}(i) = \textit{path}(j)$}
\State $j \gets j-1$.
\State $i \gets i-1$.
\State \textbf{goto} \emph{loop}.
\State \textbf{close};
\EndIf
\State $i \gets i+\max(\textit{delta}_1(\textit{string}(i)),\textit{delta}_2(j))$.
\State \textbf{goto} \emph{top}.
\EndProcedure
\end{algorithmic}
\end{algorithm}

The algorithm analyzes the destinations of all domains from all relays and if there exists at least one destination for a domain that is not in Country Y, then this increases the upperbound value.  An upperbound value of 1.0 means that every domain studied is hosted (or has a replica) outside of Country Y.  This value puts the Avoidance values in perspective for each (Country X, Country Y) pair.  

\subsection{Caveats}
The measurement methods described in Section~\ref{datasets} are not without limitations.  First, our study is solely based on IPv4 routes, which likely differ from IPv6 routes.  Here we also discuss limitations with geolocation accuracy, path asymmetry, and traceroute completeness.

\subsubsection{IPv4}
The measurements we conducted only collect and analyze IPv4 paths, and therefore all IPv6 paths are left out of our study.  IPv6 paths likely differ from IPv4 paths as not all routers that support IPv4 also support IPv6.  Future work includes studying IPv6 paths and which countries they transit, as well as a comparison of country avoidability between IPv4 and IPv6 paths. 

\subsubsection{Geolocation Accuracy}
Geolocation services and tools have been studied and proposed, and continue to be a growing research area.  We use MaxMind's geolocation service to map IP addresses to their respective countries.  While it has been shown that there are inaccuracies and incompleteness in MaxMind's data, research has also shown that other geolocation tools have similar or worse inaccuracy rates~\cite{huffaker2011geocompare}.  To address the incompleteness of the data, we cleaned up our IP to country mapping by removing all IP addresses that resulted in a `None' response when querying MaxMind.  This method provides a lowerbound on the number of countries that are included on the path, and therefore a lowerbound on the countries that can conduct surveillance.  

\subsubsection{Path Asymmetry}
Previous work has shown that paths are not symmetric most of the time -- the forward path from point A to point B does not match the reverse path from point B to point A~\cite{he2005routing}.  Most work on path asymmetry has been done at the AS level, but not at the country level.  Our measurement methods only take the forward path (from client to domain or relay) into account, and not the path from the domain or relay to the client.  

We conducted a study to measure path asymmetry at the country granularity; if country-level paths are symmetric, then the results of our measurements would be representative of the forward {\it and} reverse paths, but if the country-level paths are asymmetric, then our measurement results only provide a lowerbound on the number of countries that could potentially conduct surveillance.  Using 100 RIPE Atlas probes located around the world, and 8 Amazon EC2 instances, we ran traceroute measurements from every probe to every EC2 instance and from every EC2 instance to every probe.  After geolocating the IPs to countries, we analyzed the paths for symmetry.  First, we compared the set of countries on the forward path to the set of countries on the reverse path; this yielded about 30\% symmetry.  What we wanted to know is whether or not the reverse path has more countries on it than the forward path.  We measured how many reverse paths were a subset of the respective forward path; this was the case for 55\% of the paths.  

The results of this measurement are not convincing enough to state that country-level paths are symmetric, and therefore our measurements and results represent a lowerbound on the number of countries that transit traffic; our results are a lowerbound on how many unfavorable countries transit a client's traffic.

\subsubsection{Traceroute Accuracy and Completeness}
Our study is limited by the accuracy and completeness of traceroute.  Research has shown that there are a number of anomalies that can occur in traceroute-based measurements~\cite{augustin2006avoiding}, but most traceroute anomalies do not cause an overestimation in surveillance states.  The incompleteness of traceroutes, where a router does not respond, causes our results to be an underestimation of the number of surveillance states, and therefore also provides a lowerbound on surveillance.

\section{Nation-State Routing: Default \\Routes}
\label{measure}

\section{Measuring Country Avoidability}

\subsection{Measurement Pipeline: Avoidability with EDNS}

\subsection{Measurement Pipeline: Avoidability with Relays}

\subsection{Metrics}



\section{Preventing Transnational Detours}
\label{avoid_results}
In light of the results from analyzing where \textit{current} traffic is going in Section \ref{datasets}, we evaluate the effectiveness of open resolvers and relays for country avoidance.  We discuss our measurement methods, introduce an avoidance metric and algorithm, and present our findings.

\subsection{Measurement Pipelines}

{\bf Country Avoidance with Open Resolvers.} If content is replicated in different parts of the world, open DNS resolvers located around the globe can potentially help clients circumvent surveillance states.  Our measurement pipeline is shown in Figure \ref{??}\annie{this figure is being created}.  This measurement differs slightly from that described in Section \ref{pipeline}; instead of using RIPE Atlas probes, we query open DNS resolvers, and then traceroute from a client in Country X to the DNS responses given by the open resolvers.
\\
{\bf Country Avoidance with Relays.} Using an overlay network may help clients route around unfavorable countries or access content that is hosted in a more favorable country.  Figure \ref{fig:??}\annie{this figure is being created} shows the steps to conduct this measurement.  After selecting relay machines, we run traceroute measurements from Country X to each relay and from each relay to the set of domains. 

\subsubsection{Open Resolver and Relay Selection} 
We use eight Amazon EC2 instances, one in each geographic region (United States, Ireland, Germany, Singapore, South Korea, Japan, Australia, Brazil), as well as 4 VPS machines (France, Spain, Brazil, Singapore).  The conjunction of these two sets of machines allow us to evaluate surveillance avoidance with a geographically diverse set of relays. For an accurate comparison of country avoidance methods, we select an open resolver in each country that also has a relay in it.

\subsection{Avoidability Metrics}
\label{metrics}
We introduce a new metric and algorithm to measure how often a client in Country X can avoid a specific Country Y.  Using the proposed metric and algorithm, we can compare how well the different methods achieve country avoidance for any (Country X, Country Y) pair.

{\bf Avoidability Metric.}  This metric quantifies how often traffic can avoid Country Y when it originates in Country X.  Avoidability is explained as the fraction of paths that start in Country X and do not transit Country Y; more formally:

\[Avoidability(X,Y) = \frac{paths_{X,\bcancel{Y}}}{paths_{X}}\]

where $paths_{\bcancel{Y}}$ represent the paths from Country X that do not pass through Country Y, and $paths_{X}$ represent all paths that originate from Country X. The resulting value will be in the range [0,1], where 0 means the country is unavoidable for all of the domains in our study, and 1 means the client can avoid Country Y for all domains in our study.  This metric is used on country-level paths calculated during the measurements in Section~\ref{pipeline}.

{\bf Avoidability Algorithm with Relays.}  Measuring the avoidability of a country Y from a client in Country X using relays has two components: 1) is Country Y on the path from the client in Country X to the relay?  2) is Country Y on the path from the relay to the domain?  For every domain, our algorithm checks if there exists at least one path from the client Country X through any relay and on to the domain, and does not transit Country Y.  The psuedo-code for the algorithm is shown in Algorithm~\ref{avoid_algo}.

\begin{algorithm}[t]
\caption{Avoidability Algorithm}
\label{avoid_algo}
\begin{algorithmic}[1]
\Function{CalcAvoidance}{set $paths1$, set $paths2$, string c}
    \State set $usableRelays$
    \For{each $(relay,path)$ in $paths1$} 
    	\If{$c$ not in $path$}
		\State $usableRelays \gets path$
	\EndIf
    \EndFor
    \State set $accessibleDomains$
    \For{each $(relay,domain,path$ in $paths2$}
    \If{$relay$ in $usableRelays$}
        \If{$c$ not in $path$}
        \State $accessibleDomains \gets domain$
        \EndIf
    \EndIf
    \EndFor
    \State $D \gets$ number of all unique domains in $paths2$
    \State $A \gets$ length of $accessibleDomains$
    \State \Return $A / D$
\EndFunction
\end{algorithmic}
\end{algorithm}

The output of the algorithm is a value in the range [0,1] that can be compared to the output of the Avoidability Metric described above.  

{\bf Upperbound on Avoidability.}  While the Avoidability metric and algorithm provide a method to quantify how avoidable Country Y is from a client in Country X, it may be the case that a number of domains are hosted in Country Y, so the Avoidance value for these countries would never reach 1.0.  For this reason, we measured the upperbound on avoidance for given pair of (Country X, Country Y) that represents the best case value for avoidance.  The pseudocode for the algorithm is shown in Algorithm \ref{upperbound_algo}.

\begin{algorithm}[t]
\caption{Avoidance Upperbound Algorithm}
\label{upperbound_algo}
\begin{algorithmic}[1]
\Function{CalcUpperbound}{set $relayDomainPaths$, string $c$}
    \State $zeros(domainLocations)$
    \For{each $(r,d,p)$ in $relayDomainPaths$} 
		\State $dest \gets $ last item in $p$
		\State $domainLocations[d] \gets dest$
    \EndFor
    \State set $accessibleDomains$
    \For{each $domain$ in $domainLocations$}
    \If{$domainLocations[domain] \neq $ set[$c$]}
    \State $accessibleDomains \gets domain$
    \EndIf
    \EndFor
    \State $D \gets$ all unique domains in  $relayDomainPaths$
    \State $A \gets$ length of $accessibleDomains$
    \State \Return $A / D$
\EndFunction
\end{algorithmic}
\end{algorithm}

The algorithm analyzes the destinations of all domains from all relays and if there exists at least one destination for a domain that is not in Country Y, then this increases the upperbound value.  An upperbound value of 1.0 means that every domain studied is hosted (or has a replica) outside of Country Y.  This value puts the Avoidance values in perspective for each (Country X, Country Y) pair. 

\subsection{Results}
After applying the metrics described in Section \ref{metrics} to country-level paths, we compared avoidance values when using open resolvers, when using relays, and when using no country avoidance tool.  First, we discuss how effective open resolvers are at country avoidance.  Then we show how effective relays are at country avoidance as well as at keeping local traffic local.  Avoidance values are shown in Table \ref{tab:avoid}, where the countries we studied are shown in the top row, and the country to avoid is in the left-most column.  

\newcolumntype{d}[1]{D{.}{.}{#1}}
\begin{table*}[t]
\centering
\begin{tabular}{|P{25mm}|d{3.2}d{3.2}|d{3.2}d{3.2}|d{3.2}d{3.2}|d{3.2}d{3.2}|d{3.2}d{3.2}|}
\multicolumn{1}{l}{}    & \headrow{No Relay} & \headrow{Relays} & \headrow{No Relay} & \headrow{Relays} & \headrow{No Relay} & \headrow{Relays}   & \headrow{No Relay} & \headrow{Relays}  & \headrow{No Relay} & \headrow{Relays} \\\hline
\textit{Country}    &\multicolumn{2}{c|}{\textit{Brazil}}   &\multicolumn{2}{c|}{\textit{Netherlands}}   &\multicolumn{2}{c|}{\textit{India}} &\multicolumn{2}{c|}{\textit{Kenya}} &\multicolumn{2}{c|}{\textit{United States}}\\
\hline\hline
Brazil               &0.00     &0.00     &1.00  &1.00   &1.00    &1.00  &1.00   &1.00  &1.00  &1.00  \\\hline\hline
Canada               &.98     &1.00     &.99  &1.00   &.98    &.98  &.99   &.99  &.92  &1.00  \\\hline
United States        &\cellcolor{blue!25}.15     &\cellcolor{blue!25}.62     &\cellcolor{blue!25}.41  &\cellcolor{blue!25}.63   &\cellcolor{blue!25}.28    &\cellcolor{blue!25}.65  &\cellcolor{blue!25}.38   &\cellcolor{blue!25}.40  &\cellcolor{blue!25}0.00  &\cellcolor{blue!25}0.00  \\\hline\hline
France               &.94     &1.00     &.89  &.99   &.89    &1.00  &.77   &.98  &.89  &.99  \\\hline
Germany              &.99     &1.00     &.95  &.99   &.96    &.99  &.95   &1.00  &.99  &1.00  \\\hline
Great Britain        &.97     &1.00     &.86  &.99   &\cellcolor{blue!25}.79    &\cellcolor{blue!25}1.00  &\cellcolor{blue!25}.50   &\cellcolor{blue!25}.97  &.99  &1.00  \\\hline
Ireland              &.97     &.99     &.89  &.99   &.96    &.99  &.86   &.99  &.99  &.99  \\\hline
Netherlands          &.98     &.99     &0.00  &0.00   &.87    &.99  &\cellcolor{blue!25}.74   &\cellcolor{blue!25}.99  &.97  &.99  \\\hline
Spain                &.82     &1.00     &.99  &.99   &1.00    &1.00  &1.00   &1.00  &1.00  &1.00  \\\hline\hline
Kenya                &1.00     &1.00     &1.00  &1.00   &1.00    &1.00  &0.00   &0.00  &1.00  &1.00  \\\hline
Mauritius            &1.00     &1.00     &1.00  &1.00   &1.00    &1.00  &\cellcolor{blue!25}.67   &\cellcolor{blue!25}.99  &1.00  &1.00  \\\hline
South Africa         &1.00     &1.00     &1.00  &1.00   &1.00    &1.00  &\cellcolor{blue!25}.66   &\cellcolor{blue!25}.66  &1.00  &1.00  \\\hline\hline
United Arab Emirates &1.00     &1.00     &1.00  &1.00   &1.00    &1.00  &.84   &.99  &1.00  &1.00  \\\hline
India                &1.00     &1.00     &.99  &1.00   &0.00    &0.00  &.94   &1.00  &.99  &1.00  \\\hline
Singapore            &.99     &1.00     &.99  &1.00   &\cellcolor{blue!25}.73    &\cellcolor{blue!25}.94  &.96   &1.00  &.99  &1.00  \\\hline
\end{tabular}
\caption{Avoidance values for different techniques of country avoidance.  The upper bound on avoidance is 1.0 in most cases, but not all.  It is 
common for some European countries to host a domain, and therefore the upper bound is slightly lower than 1.0.  The upper bound on avoidance of the 
United States is significantly lower than the upper bound on avoidance for any other country; .886, .790, .844, and .765 are the upper bounds on avoidance 
of the United States for traffic originating in Brazil, Netherlands, India, and Kenya, respectively.}
\label{tab:avoid}
\end{table*}

\subsubsection{Avoidance with Open Resolvers}
\annie{This is still being calculated, but I'll update as soon as possible.}

\subsubsection{Avoidance with Relays}
As seen in Table \ref{tab:avoid}, there are two significant trends: 1) the ability for a client to avoid a given Country Y increases with the use of relays, and 2) the least avoidable countries are surveillance states.

{\bf Avoidance Increases with Relays.}
In almost every (Country X, Country Y) pair, where Country X is the client's country (Brazil, Netherlands, India, Kenya, or the United States) and Country Y is the country to avoid, the use of an overlay network makes Country Y more avoidable than the current default routes.  The one exception we encountered is when a client is located in Kenya and wants to avoid South Africa; South Africa is on the path between the client and every relay, and therefore the client should not use the relays.  

%\begin{figure}
%\centering
%\includegraphics[width=.5\textwidth]{ke_compound_new}
%\caption{Avoidance values for Kenyan clients without relays, with relays, and the upperbound.}
%\label{fig:ke_avoidance}
%\end{figure}

For 84\% of the (Country X, Country Y) pairs shown in Table \ref{tab:avoid} the avoidance with relays reaches the upper bound on avoidance.  (Kenya, Country Y) pairs have the lowest percentage of avoidance values that reach the upper bound, showing that it is more difficult for Kenyan clients to avoid a given country.  This is not to say that relays are not effective for clients in Kenya; for example, the default routes to the top 100 domains for Kenyans avoid Great Britain 50\% of the time, but with relays this percentage increases to about 98\% of the time, and the upper bound is about 98\%. Figure \ref{fig:ke_avoidance} shows default avoidance, avoidance with relays, and the upper bound for Kenya; it's clear that despite having the worst position for avoidance out of the studied countries, in most cases the avoidance with relays either reaches or because extremely close to the upper bound.  The highest percentage (100\%) of avoidance values that reach the upper bound are for clients in the United States -- relays help clients in this country avoid all other Country Y in all cases that the domain is not hosted in Country Y.  

{\bf Surveillance States are the Least Avoidable.}
Certain surveillance states discussed in Section \ref{surv} are completely unavoidable a small fraction of time from certain client locations.  France is unavoidable for a small percentage of domains for clients located in the Netherlands, Kenya, and the United States.  Similarly, clients in the Netherlands and Kenya cannot avoid Great Britain for a small fraction of domains.  

Avoidance values for (Country X, United States) pairs are significantly lower than any other Country Y for all three situations: without relays, with relays, and the upper bound.   Despite increasing clients' ability to avoid the United States, relays are not as effective at helping clients avoid this country as compared to the effectiveness of the relays at avoiding all other Country Y.  Clients in India can avoid the United States more often than clients in Brazil, Netherlands, and Kenya, with an avoidance value of .656 when using relays.  Kenyan clients can only avoid the United States 40\% of the time even while using relays.  Additionally, the upper bound for avoiding the United States is significantly lower in comparison to any other country.  

{\bf Keeping Local Traffic Local}
For the cases where there were relays located in one of the five studied countries, we evaluated how effectively the use of relays kept local traffic local.  This evaluation was possible for Brazil and the United States.  In both cases, we found the percentage of tromboning traffic decrease, and the number of countries that traffic tromboned to decreased.  Tromboning Brazilian traffic decreased from 13.2\% without relays to 9.7\% with relays; when relays are used, all tromboning traffic goes only to the United States.  With the use of relays, there was only 1.3\% tromboning traffic for a United States client, whereas without relays there was 11.2\% tromboning traffic.  For the 1.2\% of traffic that trombones from the United States, it all goes only to Ireland.

\section{Discussion}
\label{sec:discussion}

\paragraph{Avoiding multiple countries.} 
We have studied only the extent to which Internet paths can be
engineered to avoid a {single} country.  Yet, avoiding a single country
may force an Internet path into {\em other} unfavorable
jurisdictions. Future work should
explore the feasibility of avoiding multiple countries or perhaps even entire regions.


\paragraph{Evolution over time.}
Our study is based on a snapshot of paths. Over time, paths
change, hosting locations change, IXPs are built, submarine cables are
laid, and the countries conducting network interference change.  We are continuing to collect
the measurements that we have presented in this paper to facilitate future exploration
of how these characteristics evolve over time.

%\paragraph{Isolating DNS diversity vs. path diversity.}
%In our experiments, the overlay network relays perform DNS lookups from
%geographically diverse locations, which provides some level of DNS
%diversity in addition to the path diversity that the relays inherently
%provide. This approach somewhat conflates the benefits of DNS diversity
%with the benefits of path diversity and in practice may increase
%clients' vulnerability to surveillance, since each relay is performing
%DNS lookups on each client's behalf. We plan to conduct additional
%experiments where the client relies on its local DNS resolver to map
%domains to IP addresses, as opposed to relying on the relays for both
%DNS resolution and routing diversity.

\paragraph{ISPs controlling country avoidance.} 
Future work includes modifying \system{} to be implemented within an 
ISP.  Adding country avoidance functionality within ISPs 
(government-controlled or otherwise) allows ISPs to provide this as a transparent
service to customers.  A government that wishes to control which countries
its citizens' traffic is traversing might deploy \system{} in the country's ISPs.

\paragraph{Additional \system{} features.}  
The oracle could add additional steps in the decision
chain introduced in Section \ref{multiplex} that take into account
relay and path loads.  For example, if multiple relays provide a path
to a domain that does not traverse the specified country, then the
decision between the suitable proxies could be determined based on
current relay load or performance.  Our current implementation of
\system{} re-computes all paths once per five days; we could only
re-compute paths when necessary.  For example, a BGP monitoring system
detect routing changes and trigger path measurements.

%Additional features can be
%implemented at the relay to help preserve client privacy.  An example
%would be to use the relay as a mix, or to send out fake traffic to
%confuse an attacker that may be trying to perform traffic analysis at
%the relay.

\section{Related Work}
\label{related}

{\bf topic 1.}

{\bf topic 2.}

{\bf topic 3.}

\section{Conclusion}
\label{conclusion}

In this work, we have measured Internet paths to characterize routing
detours that take Internet paths through countries that perform
surveillance.  Our findings show that paths commonly traverse known
surveillance states, even when they originate and end in a
non-surveillance state.  As a possible step towards a remedy, we have
investigated how clients can use the open DNS resolver infrastructure
and overlay network relays to prevent routing detours through
unfavorable jurisdictions.  These methods give clients the power to
avoid certain countries, as well as help keep local traffic local.
Although some countries are completely avoidable, we find that some of
the more prominent surveillance states are the least avoidable.

%\end{document}  % This is where a 'short' article might terminate

%ACKNOWLEDGMENTS are optional
%\section{Acknowledgments}

%
% The following two commands are all you need in the
% initial runs of your .tex file to
% produce the bibliography for the citations in your paper.
\bibliographystyle{abbrv}
\bibliography{sigproc}  % sigproc.bib is the name of the Bibliography in this case
% You must have a proper ".bib" file
%  and remember to run:
% latex bibtex latex latex
% to resolve all references
%
% ACM needs 'a single self-contained file'!
%
%APPENDICES are optional
%\balancecolumns
\appendix
%Appendix A
\end{document}
