\section{State of Surveillance}
\label{surv}
We focused our study on five different countries, and for each, we actively measured and analyzed traffic that originated there.  These five countries were chosen for specific reasons and we present them here.  We also discuss countries that currently conduct surveillance; this exploration is not exhaustive, but highlights countries that are passing new surveillance laws and countries that have strict surveillance practices already.   

\subsection{Studied Countries}
We selected Brazil, Netherlands, Kenya, India, and the United States for the following reasons.

{\bf Brazil.} It has been widely publicized that Brazil is actively trying to avoid having their traffic transit the United States.  They have been building IXPs, deploying underwater cables to Europe, and pressuring large U.S. companies to host content within Brazil~\cite{brazil_history, brazil_break_from_US, brazil_conference,
  brazil_conference2, brazil_human_rights, brazil_cable, brazil_us_companies, brazil_IXP1}.  This effort to avoid traffic transitting a specific country led us to investigate whether their efforts have been successful or not.

{\bf Netherlands.}  We selected to study the Netherlands for two reasons: 1) the Netherlands is beginning to emerge as a site where servers are located for cloud services, such as Akamai, and 2) the Netherlands is where a large IXP is located (AMS-IX). Analyzing the Netherlands will allow us to see what effect AMS-IX and the emergence of cloud service hosting has on their traffic.

{\bf Kenya.} Prior research on the interconnectivity of Africa~\cite{gupta2014peering, fanou2015diversity} led us to explore the characterization an African country's interconnectivity.  We chose Kenya for few reasons: 1) it has more than one IXP, 2) it has high Internet access and usage (for the East African region), and 3) it is a location with many submarine cable landing points~\cite{kenya_nigeria, teams}.

{\bf India.}  India has one of the highest number of Internet users in Asia, second only to China, which has already been well-studied~\cite{tsui2003panopticon, wang2010discourse}.  With such a high number of Internet users, and presumably a large amount of Internet traffic, we study India to see where this traffic is going.

{\bf United States.}  We chose to study the United States because of how inexpensive it is to host domains there, as well as the prevalence of Internet and technology companies located there.

\subsection{Surveillance States}

%Some countries have created agreements on data sharing, while others want to control their citizens' online presence.  
When analyzing which countries Internet traffic traverse, special attention should be given to countries that may be unfavorable because of their surveillance laws.  Some of the countries that are currently conducting surveillance are the ``Five Eyes,'' ~\cite{lander2004international, eyeswideopen} the United States, Canada, United Kingdom, New Zealand, and Australia, as well as France, Germany, Poland, Hungary, Russia, Ukraine, Belarus, Kyrgyzstan, and Kazakhstan.  

{\bf Five Eyes.} The ``Five Eyes'' participants are the United States National Security Agency (NSA), the United Kingdom's Government Communications Headquarters (GCHQ), Canada's Communications Security Establishment Canada (CSEC), the Australian Signals Directorate (ASD), and New Zealand's Government Communications Security Bureau (GCSB)~\cite{eyeswideopen}.  According to the original agreement, the agencies can: 1) collect traffic; 2) acquire communications documents and equipment; 3) conduct traffic analysis; 4) conduct cryptanalysis; 5) decrypt and translate; 6) acquire information about communications organizations, procedures, practices, and equipment.  The agreement also implies that all five countries will share all intercepted material by default.  The agencies work so closely that the facilities are often jointly staffed by members of the different agencies, and it was reported ``that SIGINT customers in both capitals seldom know which country generated either the access or the product itself.''~\cite{lander2004international}.

\begin{table}[ht!]
\centering
\caption{A sample of countries that are actively conducting surveillance at different levels of intensity.}
\label{surv_table}
\begin{tabular}{|p{1.4cm}|p{1.6cm}|p{1.6cm}|p{1.6cm}|}
\hline
Survillance stage & Collecting Metadata Phone \& Internet & Demanding ISPs to Participate & No Need for Court Order \\
\hline\hline
France            & \checkmark~\cite{francesurv, francesurv2} & \checkmark~\cite{francesurv} &   \\ \hline
Germany           & \checkmark~\cite{germansurv}   &             &                              \\ \hline
Poland            & \checkmark~\cite{francesurv2}      &         &  \checkmark~\cite{francesurv2}            \\ \hline
Hungry            & \checkmark~\cite{francesurv2}        &             & \checkmark~\cite{francesurv2} \\ \hline
Russia            & \checkmark~\cite{francesurv2}    & \checkmark~\cite{russiasurv, russiasurv2}    &  \\ \hline
%....              &      &        &   \\    \hline
\end{tabular}
\end{table}

A number of other countries are passing laws to facilitate mass surveillance.  These laws have differing levels of intensity, which can be seen in Table \ref{surv_table}; the countries with the least intense surveillance laws are listed at the top of the table, and those with the more intense laws are listed at the bottom.  These countries, along with the ``Five Eyes'' participants should be flagged when characterizing transnational detours in the following section.

%{\bf France.} France recently passed a new surveillance law that authorizes the government closely monitor the mobile phone and Internet communications of French citizens.  Additionally, the law requires ISPs to install ``black boxes'' that are designed to collect and analyze metadata on the Internet usage of millions of people.  The law allows surveillance without much oversight and the conditions under which the law's powers can be used are vague~\cite{francesurv}. The French Parliament has also made it easier for the government to access encrypted data for criminal investigations by enhancing penalties against companies that refuse to cooperate~\cite{francesurv2}.

%{\bf Germany.} A new data retention law was recently passed in Germany that will allow law enforcement agencies to access metadata of phone calls and Internet connections~\cite{germansurv}.  Prior to the bill, telecom providers only retained data for business purposes, but the bill will modify the German Telecommunications Act and require telecom providers to retain traffic data on phone calls and Internet connections.  The data collected includes IP addresses of users as well as the date and time of Internet connections.

%{\bf Poland.}  Poland now has stricter surveillance laws by ammending the Police Act.  This gives Polish authorities access to metadata without court approval~\cite{francesurv2}.

%{\bf Hungary.}  Hungary has pushed surveillance even further by proposing a new bill that makes it a crime for service providers to use encryption-based application or software.  It also forces ISPs to build back doors that give the government access to data.  Under the bill, ISPs are required to collect metadata on anyone who uses encryption, and it criminalizes the refusal of companies to give collected data to the government~\cite{francesurv2}.

%{\bf Russia.} Russia has had a surveillance infrastructure for years: SORM (System of Operative-Investigative Measures).  SORM involves a series of black boxes that ISPs must install; the providers are required to pay for the SORM equipment and installation and they are denied access to the box~\cite{russiasurv}.  If an ISP fails to cooperate, it is fined, and if the violation persists, then its license may be revoked.  Former Soviet Republics have followed in Russia's footsteps: Belarus, Ukraine, and Kyrgyzstan have adopted national interception systems modeled after SORM~\cite{russiasurv}.  Many popular sites, such as Facebook, Google, and Twitter are not hosted in Russia, which challenges surveillance.  In the past year, Russia has passed a data localization law.  This law states that any company that collects personal information from Russian users must stor their data on servers within the country - the main targets of the law were Facebook, Google, and Twitter~\cite{russiasurv2}. 
