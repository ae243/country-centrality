\section{State of Surveillance}
\label{surv}

\subsection{Studied Countries}

\subsection{Vantage Points}
To our knowledge, the publicly available traceroute datasets suitable for our goal are from iPlane~\cite{madhyastha2006iplane} and CAIDA (Center for Applied Internet Data Analysis)~\cite{caida}.  The iPlane project uses PlanetLab ~\cite{planetlab} nodes to run traceroutes to a random set of IP addresses that cover all BGP atoms.  This project also has historical data as far back as 2006.  Unfortunately, because iPlane uses PlanetLab nodes, which have been shown to mostly use the Global Research and Education Network (GREN), the traceroutes run from PlanetLab nodes will not be representative of typical Internet users' traffic paths~\cite{banerjee2004interdomain}.  The other publicly available dataset, from CAIDA, ran traceroutes from different vantage points around the world, but to randomized destination IP addresses that cover all /24s.  This is also not sufficient for what we wanted to measure because a typical Internet user is going to access a domain that will be locally resolved; the user will not input a specific IP address in their browser.  Therefore, we chose to run active measurements that would be most representative of an Internet user.  We chose to run DNS and traceroute measurements from RIPE Atlas probes, which are hosted all around the world and in many different settings, include home networks~\cite{ripe_atlas}.  The first advantage of using RIPE Atlas is that the probes can use the local DNS resolver, which would give us the best estimate of the traceroute destination.  The next advantage is that we can control the parameters of traceroute; we specified to use paris traceroute as well as conducting both ICMP and TCP traceroutes.  We discuss more of our parameter selection and methodology in Section~\ref{measure}.

Our study looks at the Alexa Top 100 domains in different countries, as well as the 3rd party domains that are requested as part of an original web request.  To obtain these 3rd party domains we {\tt curl} each of the Top 100 domains, but we must do so from within the country we are studying.  There is no current functionality to {\tt curl} from RIPE Atlas probes, so we established a VPN connection within each of these countries to {\tt curl} each domain and extract the 3rd party domains.

As one of our research goals is to quantify how well the use of an overlay network will help clients avoid surveillance, we need access to a set of relays.  We used eight Amazon EC2 instances, one in each geographic region (United States, Ireland, Germany, Singapore, South Korea, Japan, Australia, Brazil), as well as 4 VPS machines (France, Spain, Brazil, Singapore).  The conjunction of these two sets of machines allow us to evaluate surveillance avoidance with a geographically diverse set of relays.

\subsection{Surveillance States}

Some countries have created agreements on data sharing, while others want to control their citizens' online presence.  Some of the countries that are currently conducting surveillance are the ``Five Eyes,'' the United States, Canada, United Kingdom, New Zealand, and Australia, France, Germany, Poland, Hungary, Russia, Ukraine, Belarus, Kyrgyzstan, and Kazakhstan.  While already fairly long, this list is not exhaustive, and merely points out some of the most active surveillance states.

{\bf Five Eyes.} The ``Five Eyes'' participants are the United States National Security Agency (NSA), the United Kingdom's Government Communications Headquarters (GCHQ), Canada's Communications Security Establishment Canada (CSEC), the Australian Signals Directorate (ASD), and New Zealand's Government Communications Security Bureau \\(GCSB)~\cite{eyeswideopen}.  According to the original agreement, the agencies can: 1) collect traffic; 2) acquire communications documents and equipment; 3) conduct traffic analysis; 4) conduct cryptanalysis; 5) decrypt and translate; 6) acquire information about communications organizations, procedures, practices, and equipment.  The agreement also implies that all five countries will share all intercepted material by default.  The agencies work so closely that the facilities are often jointly staffed by members of the different agencies, and it was reported ``that SIGINT customers in both capitals seldom know which country generated either the access or the product itself.''~\cite{lander2004international}.

{\bf France.} France recently passed a new surveillance law that authorizes the government closely monitor the mobile phone and Internet communications of French citizens.  Additionally, the law requires ISPs to install ``black boxes'' that are designed to collect and analyze metadata on the Internet usage of millions of people.  The law allows surveillance without much oversight and the conditions under which the law's powers can be used are vague~\cite{francesurv}. The French Parliament has also made it easier for the government to access encrypted data for criminal investigations by enhancing penalties against companies that refuse to cooperate~\cite{francesurv2}.

{\bf Germany.} A new data retention law was recently passed in Germany that will allow law enforcement agencies to access metadata of phone calls and Internet connections~\cite{germansurv}.  Prior to the bill, telecom providers only retained data for business purposes, but the bill will modify the German Telecommunications Act and require telecom providers to retain traffic data on phone calls and Internet connections.  The data collected includes IP addresses of users as well as the date and time of Internet connections.

{\bf Poland.}  Poland now has stricter surveillance laws by ammending the Police Act.  This gives Polish authorities access to metadata without court approval~\cite{francesurv2}.

{\bf Hungary.}  Hungary has pushed surveillance even further by proposing a new bill that makes it a crime for service providers to use encryption-based application or software.  It also forces ISPs to build back doors that give the government access to data.  Under the bill, ISPs are required to collect metadata on anyone who uses encryption, and it criminalizes the refusal of companies to give collected data to the government~\cite{francesurv2}.

{\bf Russia.} Russia has had a surveillance infrastructure for years: SORM (System of Operative-Investigative Measures).  SORM involves a series of black boxes that ISPs must install; the providers are required to pay for the SORM equipment and installation and they are denied access to the box~\cite{russiasurv}.  If an ISP fails to cooperate, it is fined, and if the violation persists, then its license may be revoked.  Former Soviet Republics have followed in Russia's footsteps: Belarus, Ukraine, and Kyrgyzstan have adopted national interception systems modeled after SORM~\cite{russiasurv}.  Many popular sites, such as Facebook, Google, and Twitter are not hosted in Russia, which challenges surveillance.  In the past year, Russia has passed a data localization law.  This law states that any company that collects personal information from Russian users must stor their data on servers within the country - the main targets of the law were Facebook, Google, and Twitter~\cite{russiasurv2}. 
