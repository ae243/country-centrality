\section{Introduction}
\label{intro}

When Internet traffic enters a country, it becomes subject to that
country's laws.  As a result, users have more need than ever to
determine---and control---which countries their traffic is traversing.
An increasing number of countries have passed laws that facilitate mass
surveillance of their networks~\cite{france_surveillance,
  netherlands_surveillance, kazak_surveillance, uk_bill}, and governments
and citizens are increasingly motivated to divert their Internet traffic
from countries that perform surveillance (notably, the United
States~\cite{russia_secure_internet,
  routing_errors, dte}).

Many countries---notably, Brazil---are taking impressive measures to
reduce the likelihood that Internet traffic transits the United
States~\cite{brazil_history, brazil_break_from_US, brazil_conference,
  brazil_conference2, brazil_human_rights} including building a
3,500-mile long fiber-optic cable from Fortaleza to Portugal (with no
use of American vendors); pressing companies such as Google, Facebook,
and Twitter (among others) to store data locally; and mandating the
deployment of a state-developed email system (Expresso) throughout the
federal government (instead of what was originally used, Microsoft
Outlook)~\cite{brazil_cable, brazil_us_companies}.  Brazil is also
building Internet Exchange Points (IXPs)~\cite{brazil_IXP1}, now has the
largest national ecosystem of public IXPs in the
world~\cite{brazil_ixp_ecosystem}, and the number of internationally
connected Autonomous Systems (ASes) continues to
grow~\cite{brazil_international_ases}. Brazil is not alone: IXPs are
proliferating in eastern Europe, Africa, and other regions, in part out
of a desire to ``keep local traffic local''. Building IXPs alone, of
course, cannot guarantee that Internet traffic for some service does not
enter or transit a particular country: Internet protocols have no notion
of national borders, and interdomain paths depend in large part on
existing interconnection business relationships (or lack thereof).

Although end-to-end encryption stymies surveillance by
concealing URLs and content, it does not by itself protect all sensitive
information from prying eyes. First, many websites do not fully support
encrypted browsing by default; a recent study showed that more than 85\%
of the most popular health, news, and shopping sites do not encrypt by
default~\cite{what_isps_can_see}; migrating a website to HTTPS is
challenging doing so requires all third-party domains on the site
(including advertisers) to use HTTPS.  Second, even encrypted traffic
may still reveal a lot about user behavior: the presence of any
communication at all may be revealing, and website fingerprinting can
reveal information about content merely based on the size, content, and
location of third-party resources that a client loads. DNS traffic is
also quite revealing and is essentially never
encrypted~\cite{what_isps_can_see}.  Third, ISPs often terminate TLS
connections, conducting man-in-the-middle attacks on encrypted traffic
for network management purposes~\cite{mitm_isp}. Circumventing
surveillance thus requires not only encryption, but also mechanisms
for controlling where traffic goes in the first place.

In this paper, we study two questions: (1)~Which countries do {\em
  default} Internet routing paths traverse?; (2)~What methods can help
increase hosting and path diversity to help governments and citizens
better control transnational Internet paths?  In contrast to previous
work~\cite{karlin2009nation}, which simulates Internet paths, we
\textit{actively measure} and analyze the paths originating in five
different countries: Brazil, Netherlands, Kenya, India, and the United
States.  We study these countries for different reasons, including their
efforts made to avoid certain countries, efforts in building out IXPs,
and their low cost of hosting domains.  Our work studies the
router-level forwarding path, which differs from all other work in this
area, which has focused on analyzing Border Gateway Protocol
(BGP) routes~\cite{karlin2009nation,shah2015characterizing}.  Although
BGP routing can offer useful information about paths, it does not
necessarily reflect the path that traffic actually takes, and it only
provides AS-level granularity, which is often too coarse to make strong
statements about which countries that traffic is traversing.  In
contrast, we measure traffic routes from RIPE Atlas probes in five
countries to the Alexa Top 100 domains for each country; we directly
measure the paths not only to the websites corresponding to the
themselves, but also to the sites hosting any third-party content on
each of these sites.

Determining which countries a client's traffic traverses is challenging, for
several reasons.  First, performing direct measurements is more costly
than passive analysis of BGP routing tables; RIPE Atlas, in particular,
limits the rate at which one can perform measurements.  As a result, we
had to be strategic about the origins and destinations that we selected
for our study. As we explain in Section~\ref{surv}, we study five
geographically diverse countries, 
focusing on countries in each region that are
making active attempts to thwart transnational Internet paths.  Second,
IP geolocation---the process of determining the geographic location of an
IP address---is notoriously challenging, particularly for IP addresses
that represent Internet infrastructure, rather than end-hosts. We cope
with this inaccuracy by making conservative estimates of the extent of
routing detours, and by recognizing that our goal is not to pinpoint a
precise location for an IP address as much as to achieve accurate
reports of {\em significant} off-path detours to certain countries or
regions. (Section~\ref{datasets} explains our method in more detail; we
also explicitly highlight ambiguities in our results.) Finally, the
asymmetry of Internet paths can also make it difficult to analyze the
countries that traffic traverses on the reverse path from server to
client; our study finds that country-level paths are often asymmetric,
and, as such, our findings represent a lower bound on transnational
routing detours.

The first part of our study (Section~\ref{datasets}) characterizes the
current state of transnational Internet routing detours.  We first
explore hosting diversity and find that only about half of the Alexa Top
100 domains in the five countries studied are hosted in more than one
country, and many times that country is a surveillance state that
clients may want to avoid. Second, even if hosting diversity can be
improved, routing can still force traffic through a small
collection of countries (often surveillance states). Despite strong
efforts made by some countries to ensure their traffic does not transit
unfavorable countries~\cite{brazil_history, brazil_break_from_US,
  brazil_conference, brazil_conference2, brazil_human_rights}, their
traffic still traverses surveillance states.  Over 50\% of the top
domains in Brazil and India are hosted in the United States, and over
50\% of the paths from the Netherlands to the top domains transit the
United States.  About half of Kenyan paths to the top domains traverse
the United States and Great Britain (but the same half does not traverse
both countries).  Much of this phenomenon is due to ``tromboning'',
whereby an Internet path starts and ends in a country, yet transits an
intermediate country; for example, about 13\% of the paths that we
explored from Brazil tromboned through the United States.
Infrastructure building alone is not enough: ISPs in respective regions
need better encouragements to interconnect with one another to ensure
that local traffic stays local.

The second part of our work (Section~\ref{avoid_results}) explores
potential mechanisms for avoiding certain countries, and the potential
effectiveness of these techniques.  We explore two techniques: using the
open DNS resolver infrastructure and using overlay network relays.  We
find that both of these techniques can be effective for clients in
certain countries, yet the effectiveness of each technique also depends
on the county.  For example, Brazilian clients could completely avoid
Spain, Italy, France, Great Britain, Argentina, and Ireland (among
others), even though the default paths to many popular Brazilian sites
traverse these countries. Additionally, overlay network relays can keep
local traffic local: by using relays in the client's country, fewer
paths trombone out of the client's country.  The percentage of
tromboning paths from the United States decreases from 11.2\% to 1.3\%
when clients take advantage of a small number of overlay network relays.

We also find that some of the most prominent surveillance states are
also some of the least avoidable countries.  For example, many countries
depend on ISPs in the United States, a known surveillance state, for
connectivity to popular sites and content.  Neither Brazil, India, Kenya, nor
the Netherlands must traverse the United States to reach many of the
popular local websites, even if they use open resolvers and network
relays. Using overlay network relays, both Brazilian and Netherlands
clients can avoid the United States for about 65\% of paths; yet, the
United States is completely unavoidable for about 10\% of the paths
because it is the only country where the content is hosted.  Kenyan
clients can only avoid the United States on about 55\% of the paths.  On
the other hand, the United States can avoid every other country except
for France and the Netherlands, and even then they are avoidable for
99\% of the top domains.
