\section{Measurement Infrastructure}
\label{datasets}
For this study, we made careful choices about the measurement infrastructure.  These included which vantage points within a country to use, which servers to use as relays, and which method to use for geolocating IP addresses.  We also introduce our measurement methods for measuring where current Internet traffic is traveling to, how well open DNS resolvers help clients avoid specific countries, and how well relays help clients avoid specific countries.

\subsection{Resources}
\subsubsection{Vantage Points}
To our knowledge, the publicly available traceroute datasets suitable for our goal are from iPlane~\cite{madhyastha2006iplane} and CAIDA (Center for Applied Internet Data Analysis)~\cite{caida}.  The iPlane project uses PlanetLab ~\cite{planetlab} nodes to run traceroutes to a random set of IP addresses that cover all BGP atoms.  This project also has historical data as far back as 2006.  Unfortunately, because iPlane uses PlanetLab nodes, which have been shown to mostly use the Global Research and Education Network (GREN), the traceroutes run from PlanetLab nodes will not be representative of typical Internet users' traffic paths~\cite{banerjee2004interdomain}.  The other publicly available dataset, from CAIDA, ran traceroutes from different vantage points around the world, but to randomized destination IP addresses that cover all /24s.  This is also not sufficient for what we wanted to measure because a typical Internet user is going to access a domain that will be locally resolved; the user will not input a specific IP address in their browser.  Therefore, we chose to run active measurements that would be most representative of an Internet user.  We chose to run DNS and traceroute measurements from RIPE Atlas probes, which are hosted all around the world and in many different settings, include home networks~\cite{ripe_atlas}.  The first advantage of using RIPE Atlas is that the probes can use the local DNS resolver, which would give us the best estimate of the traceroute destination.  The next advantage is that we can control the parameters of traceroute; we specified to use paris traceroute as well as conducting both ICMP and TCP traceroutes.  We discuss more of our parameter selection and methodology in Section~\ref{measure}.

Our study looks at the Alexa Top 100 domains in different countries, as well as the 3rd party domains that are requested as part of an original web request.  To obtain these 3rd party domains we {\tt curl} each of the Top 100 domains, but we must do so from within the country we are studying.  There is no current functionality to {\tt curl} from RIPE Atlas probes, so we established a VPN connection within each of these countries to {\tt curl} each domain and extract the 3rd party domains.

\subsubsection{Servers}
As one of our research goals is to quantify how well the use of an overlay network will help clients avoid surveillance, we need access to a set of relays.  We used eight Amazon EC2 instances, one in each geographic region (United States, Ireland, Germany, Singapore, South Korea, Japan, Australia, Brazil), as well as 4 VPS machines (France, Spain, Brazil, Singapore).  The conjunction of these two sets of machines allow us to evaluate surveillance avoidance with a geographically diverse set of relays.

\subsubsection{Geolocation Data}
Previous work has shown that there are fundamental challenges in deducing a geographic location from an IP address, despite using different methods such as DNS names of the target, network delay measurements, and host-to-location mapping in conjunction with BGP prefix information~\cite{padmanabhan2001investigation}.  Because the focus of this work is on measuring and avoiding surveillance, and not on geolocation algorithms, we used a pre-existing geolocation service: MaxMind~\cite{maxmind}.

\subsection{Measurement Pipeline: Current Traffic}
\label{pipeline1}
Our methods for measuring where traffic paths go use the data-plane; we analyze the reported hops of traceroute measurements to find which countries are on the path from a client in Country X to a popular domain.  For this study, we used RIPE Atlas probes in Country X, specifically the set of probes that had unique ASes in the country.  The destinations for traceroutes are the Country X Alexa Top 100 domains, as well as the third party domains within the response bodies of the 100 domains.  There are three main steps to our measurement procedure: traceroute generation and collection, transformation of traceroutes to country-level paths, and path analysis.

{\bf Step 1.} The first step is to {\tt curl} each of the Country X Alexa Top 100 domains, and extract the third party domains from the response body.  These third party domains will each be fetched by the client when the original webpage is rendered in the browser.  Next, the RIPE Atlas probes in Country X will locally resolve each domain (and third party domain) by running a DNS measurement.  The local resolution is representative of a DNS response that an Internet user in Brazil would receive.  Once the DNS responses are received for all DNS measurements, the IP addresses are consolidated into a list of /24 subnets that cover the set of IP addresses; this is done because all IP addresses in a /24 network should geolocate to the same country.  The last part of this step is running traceroutes from the RIPE Atlas probes to a single IP address in each of the /24 subnets calculated from the DNS responses.  The measurements were run using paris traceroute and each (probe, destination IP) pair was used twice: once using ICMP traceroute and once using TCP traceroute.  

{\bf Step 2.}  Step 2 generates country-level paths from the set of traceroutes produced during Step 1.  Using MaxMind, each IP address was geolocated at a country granularity.  This resulted in a set of country-level paths.

{\bf Step 3.}  After calculating country-level paths, we perform different data analysis to find where traffic travels.  We look for countries that are on the path between the client and the destination.  These countries have the potential to conduct surveillance on the traffic of clients in Country X.  A different analysis can show the countries in which the path ends, representing where the content is hosted.  It's important to note that the destinations found in this study could change if the content is georeplicated in other countries.  On the otherhand, if it is not georeplicated, then the destination's country is unavoidable.  We study this more closely in Section \ref{metrics}.  Lastly, we look at how much tromboning occurs - a tromboning path is one that starts and ends in Country X, but also traverses at least one other country.  These paths are interesting because they should not be exiting the country if the destination is in the country of origin.  

\subsection{Measurement Pipeline: Country Avoidance with Open Resolvers}
\label{pipeline2}
If content is replicated in different parts of the world, open DNS resolvers located around the globe can potentially help clients circumvent surveillance states.  To measure this, we first pick a set of open resolvers to query, then query with Country X's Alexa Top 100 domains, and traceroute from a client in Country X to the DNS responses given by the open resolvers.  These three steps are explained in more detail below.

{\bf Step 1.}  Similar to the method in Section~\ref{pipeline1}, the first step is to {\tt curl} each of the Country X Alexa Top 100 domains, and extract the third party domains from the response body.  Next, a set of open DNS resolvers are selected that are geographically diverse; the set includes resolvers 10 different countries -- the same countries in which the relays (servers) are located (United States, Brazil, Ireland, France, Germany, Spain, Singapore, South Korea, Japan, Australia).  We query each open resolver for every domain and third party domain that was collected.

{\bf Step 2.}  After receiving the set of IP address from DNS responses, a VPN connection to Country X is established.  Through this connection, traceroute measurements are run to every IP address returned by the open resolvers. The result is a set of IP-level paths, which are then geolocated using MaxMind, and provides a set of country-level paths.

{\bf Step 3.}  After mapping the IP-level paths to country-level paths, we analyze the countries on each path.  We look for known surveillance states that are on the path from the client to the destination, where the content is hosted, and which countries are unavoidable.  We compare these results to those of the current state of nation-state routing to determine if open resolvers improve country avoidability.

\subsection{Measurement Pipeline: Country Avoidance with Relays}
Using an overlay network may help clients route around unfavorable countries or access content that is hosted in a more favorable country.  First, we select relays that are geographically diverse, run traceroute measurements from Country X to each relay and from each relay to the Country X Alexa Top 100 domains.  Based on these two sets of traceroutes, we can run analyses to determine if they improve country avoidability.

{\bf Step 1.}  First, we select a set of relays; there is at least one relay in each country that we used an open resolver in: United States, Brazil, Ireland, France, Germany, Spain, Singapore, South Korea, Japan, and Australia.  Next, a VPN connection to Country X is established, and traceroute measurements are run to each relay.

{\bf Step 2.}  The second step starts with {\tt curl}ing each of the Country X Alexa Top 100 domains, and extracting the third party domains from the response bodies.  After establishing an {\tt ssh} connection to each relay, we run traceroute measurements to each of the domains and third party domains.  At this point, we have two sets of traceroutes: 1) from Country X to all relays, and 2) from all relays to all domains.

{\bf Step 3.}  Like the previous measurement pipelines, the traceroutes are mapped from IP-level paths to country-level paths using MaxMind.

{\bf Step 4.}  The last step is analyzing the two sets for country avoidance.  We measure which countries are avoidable, how often they are avoidable, and how much better relays can achieve country avoidance in comparison to open resolvers.

\subsection{Avoidability Metrics}
\label{metrics}
We introduce a new metric and algorithm to measure how often a client in Country X can avoid a specified Country Y.  Using the proposed metric and algorithm, we can compare how well the different methods achieve country avoidance for a specified client Country X and unfavorable Country Y.

{\bf Avoidability Metric.}  This metric quantifies how often traffic can avoid Country Y when it originates in Country X.  Avoidability is explained as the fraction of paths that start in Country X and do not transit Country Y; more formally:

\[Avoidability(X,Y) = \frac{paths_{X,\bcancel{Y}}}{paths_{X}}\]

where $paths_{\bcancel{Y}}$ represent the paths from Country X that do not pass through Country Y, and $paths_{X}$ represent all paths that originate from Country X. The resulting value will be in the range [0,1], where 0 means the country is unavoidable for all of the domains in our study, and 1 means the client can avoid Country Y for all domains in our study.  This metric is used on country-level paths calculated during the measurements in Sections~\ref{pipeline1} and \ref{pipeline2}.  \annie{This may need to be updated to include domain weights based on the power law}

{\bf Avoidability Algorithm with Relays.}  Measuring the avoidability of a country Y from a client in Country X using relays has two components: 1) is Country Y on the path from the client in Country X to the relay?  2) is Country Y on the path from the relay to the domain?  For every domain, our algorithm checks if there exists at least one path from the client Country X through any relay and on to the domain, and does not transit Country Y.  The psuedo-code for the algorithm is shown in Algorithm~\ref{avoid_algo}.

\begin{algorithm}
\caption{My algorithm}\label{avoid_algo}
\begin{algorithmic}[1]
\Procedure{MyProcedure}{}
\State $\textit{stringlen} \gets \text{length of }\textit{string}$
\State $i \gets \textit{patlen}$
\BState \emph{top}:
\If {$i > \textit{stringlen}$} \Return false
\EndIf
\State $j \gets \textit{patlen}$
\BState \emph{loop}:
\If {$\textit{string}(i) = \textit{path}(j)$}
\State $j \gets j-1$.
\State $i \gets i-1$.
\State \textbf{goto} \emph{loop}.
\State \textbf{close};
\EndIf
\State $i \gets i+\max(\textit{delta}_1(\textit{string}(i)),\textit{delta}_2(j))$.
\State \textbf{goto} \emph{top}.
\EndProcedure
\end{algorithmic}
\end{algorithm}

The output of the algorithm is a value in the range [0,1] that can be compared to the output of the Avoidability Metric described above.  

{\bf Upperbound on Avoidability.}  While the Avoidability metric and algorithm provide a method to quantify how avoidable Country Y is from a client in Country X, it may be the case that a number of domains are hosted in Country Y, so the Avoidance value for these countries would never reach 1.0.  For this reason, we measured the upperbound on avoidance for given pair of (Country X, Country Y) that represents the best case value for avoidance.  The pseudocode for the algorithm is shown in Algorithm \ref{upperbound_algo}.

\begin{algorithm}
\caption{My algorithm}\label{upperbound_algo}
\begin{algorithmic}[1]
\Procedure{MyProcedure}{}
\State $\textit{stringlen} \gets \text{length of }\textit{string}$
\State $i \gets \textit{patlen}$
\BState \emph{top}:
\If {$i > \textit{stringlen}$} \Return false
\EndIf
\State $j \gets \textit{patlen}$
\BState \emph{loop}:
\If {$\textit{string}(i) = \textit{path}(j)$}
\State $j \gets j-1$.
\State $i \gets i-1$.
\State \textbf{goto} \emph{loop}.
\State \textbf{close};
\EndIf
\State $i \gets i+\max(\textit{delta}_1(\textit{string}(i)),\textit{delta}_2(j))$.
\State \textbf{goto} \emph{top}.
\EndProcedure
\end{algorithmic}
\end{algorithm}

The algorithm analyzes the destinations of all domains from all relays and if there exists at least one destination for a domain that is not in Country Y, then this increases the upperbound value.  An upperbound value of 1.0 means that every domain studied is hosted (or has a replica) outside of Country Y.  This value puts the Avoidance values in perspective for each (Country X, Country Y) pair.  

\subsection{Caveats}
The measurement methods described in Section~\ref{datasets} are not without limitations.  First, our study is solely based on IPv4 routes, which likely differ from IPv6 routes.  Here we also discuss limitations with geolocation accuracy, path asymmetry, and traceroute completeness.

\subsubsection{IPv4}
The measurements we conducted only collect and analyze IPv4 paths, and therefore all IPv6 paths are left out of our study.  IPv6 paths likely differ from IPv4 paths as not all routers that support IPv4 also support IPv6.  Future work includes studying IPv6 paths and which countries they transit, as well as a comparison of country avoidability between IPv4 and IPv6 paths. 

\subsubsection{Geolocation Accuracy}
Geolocation services and tools have been studied and proposed, and continue to be a growing research area.  We use MaxMind's geolocation service to map IP addresses to their respective countries.  While it has been shown that there are inaccuracies and incompleteness in MaxMind's data, research has also shown that other geolocation tools have similar or worse inaccuracy rates~\cite{huffaker2011geocompare}.  To address the incompleteness of the data, we cleaned up our IP to country mapping by removing all IP addresses that resulted in a `None' response when querying MaxMind.  This method provides a lowerbound on the number of countries that are included on the path, and therefore a lowerbound on the countries that can conduct surveillance.  

\subsubsection{Path Asymmetry}
Previous work has shown that paths are not symmetric most of the time -- the forward path from point A to point B does not match the reverse path from point B to point A~\cite{he2005routing}.  Most work on path asymmetry has been done at the AS level, but not at the country level.  Our measurement methods only take the forward path (from client to domain or relay) into account, and not the path from the domain or relay to the client.  

We conducted a study to measure path asymmetry at the country granularity; if country-level paths are symmetric, then the results of our measurements would be representative of the forward {\it and} reverse paths, but if the country-level paths are asymmetric, then our measurement results only provide a lowerbound on the number of countries that could potentially conduct surveillance.  Using 100 RIPE Atlas probes located around the world, and 8 Amazon EC2 instances, we ran traceroute measurements from every probe to every EC2 instance and from every EC2 instance to every probe.  After geolocating the IPs to countries, we analyzed the paths for symmetry.  First, we compared the set of countries on the forward path to the set of countries on the reverse path; this yielded about 30\% symmetry.  What we wanted to know is whether or not the reverse path has more countries on it than the forward path.  We measured how many reverse paths were a subset of the respective forward path; this was the case for 55\% of the paths.  

The results of this measurement are not convincing enough to state that country-level paths are symmetric, and therefore our measurements and results represent a lowerbound on the number of countries that transit traffic; our results are a lowerbound on how many unfavorable countries transit a client's traffic.

\subsubsection{Traceroute Accuracy and Completeness}
Our study is limited by the accuracy and completeness of traceroute.  Research has shown that there are a number of anomalies that can occur in traceroute-based measurements~\cite{augustin2006avoiding}, but most traceroute anomalies do not cause an overestimation in surveillance states.  The incompleteness of traceroutes, where a router does not respond, causes our results to be an underestimation of the number of surveillance states, and therefore also provides a lowerbound on surveillance.
