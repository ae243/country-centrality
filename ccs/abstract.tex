\begin{abstract}
An increasing number of countries are passing laws that facilitate the
mass surveillance of their citizens. In response, governments and
citizens are increasingly paying attention to the countries that their
Internet traffic traverses. In some cases, countries are taking extreme
steps, such as building new IXPs and encouraging local interconnection
to keep local traffic local. We find that although many of these efforts
are extensive, they are often futile, due to the inherent lack of
hosting and route diversity for many popular sites. First, we characterize 
transnational routing detours by measuring the country-level paths to 
popular domains.  Then, we investigate how
the use of overlay network relays and the DNS open resolver
infrastructure can prevent traffic from traversing certain jurisdictions. 
We find that current traffic is traversing known surveillance states, but 
also show that there are tools that can be used for country avoidance. Our 
results show that 84\% of paths originating in Brazil traverse the United States, 
but when relays are used for country avoidance, only 37\% of Brazilian paths 
traverse the United States.  Unfortunately, we also find that 
some of the more prominent surveillance states are also some of the least avoidable 
countries.
\end{abstract}
