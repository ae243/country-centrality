\section{State of Surveillance and Interference}
\label{surv}

We focused on traffic {\em originating} in five countries:

\paragraph{Brazil.} Brazil is actively trying to avoid having their
traffic transit the U.S. They have been building IXPs,
deploying underwater cables to Europe, and pressuring large
U.S. companies to host content within Brazil~\cite{brazil_history,
  brazil_break_from_US, brazil_conference, 
  brazil_conference2, brazil_human_rights, brazil_cable,
  brazil_us_companies, brazil_IXP1}.  These efforts to avoid 
a specific country led us to investigate whether they
have been successful. 

\paragraph{Netherlands.}  First, the Netherlands is beginning to emerge
as a site where servers are located for cloud services, such as Akamai.
Second, the Netherlands is where a large IXP is located (AMS-IX). Third, they
are drafting a mass surveillance
law~\cite{netherlands_surveillance}. Analyzing the Netherlands will
allow us to see what effect AMS-IX and the emergence of cloud service
hosting has had on traffic. 

\paragraph{Kenya.} Previous research on the interconnectivity of
Africa~\cite{gupta2014peering, fanou2015diversity} led us to explore the
characterization of an African country's interconnectivity.  We chose
Kenya for three reasons: 1) it terminates many submarine cable landings;
2) it has relatively high Internet access and usage; and 3) it has more
than one IXP~\cite{kenya_nigeria, teams}. 

\paragraph{India.}  India has one of the highest number of Internet
users in Asia, second only to China, which has already been
well-studied~\cite{tsui2003panopticon, wang2010discourse}.  

\paragraph{United States.}  We chose to study the United States because
of how inexpensive it is to host domains there, the prevalence of
Internet and technology companies located there, and because it is a
known surveillance state. 

When analyzing which countries Internet traffic {\em traverses}, we gave
additional attention to countries that have known laws and practices
inolving surveillance of or interference with Internet traffic.
These countries include, the ``Five Eyes''
~\cite{lander2004international, eyeswideopen} (the United States,
Canada, United Kingdom, New Zealand, and Australia), as well as France,
Germany, Poland, Hungary, Russia, Ukraine, Belarus, Kyrgyzstan, and
Kazakhstan.  Countries such as China, Iran, and
Russia, are censoring, blocking, and interfering with traffic that
crosses their borders.  We have studied surveillance states in more
detail, and that information is available in our technical report~\cite{characterizing_detours}. 

%\paragraph{Five Eyes.} The ``Five Eyes'' participants are the United States National Security Agency (NSA), the United Kingdom's Government Communications Headquarters (GCHQ), Canada's Communications Security Establishment Canada (CSEC), the Australian Signals Directorate (ASD), and New Zealand's Government Communications Security Bureau (GCSB)~\cite{eyeswideopen}.  According to the original agreement, the agencies can: 1) collect traffic; 2) acquire communications documents and equipment; 3) conduct traffic analysis; 4) conduct cryptanalysis; 5) decrypt and translate; 6) acquire information about communications organizations, procedures, practices, and equipment.  The agreement also implies that all five countries will share all intercepted material by default.  The agencies work so closely that the facilities are often jointly staffed by members of the different agencies, and it was reported ``that SIGINT customers in both capitals seldom know which country generated either the access or the product itself.''~\cite{lander2004international}.

%\begin{table}[t!]
%\centering
%\begin{small}
%\resizebox{\columnwidth}{!}{%
%\begin{tabular}{|p{2cm}|p{1.6cm}p{1.6cm}p{1.6cm}p{1.6cm}|}
%\hline
% & Collecting Metadata (Phone, Internet) & Requiring ISPs to Participate & No Need for Court Order & Targeted Surveillance \\
%\hline
%France            & \checkmark~\cite{francesurv, francesurv2} & \checkmark~\cite{francesurv} &  & \\ 
%Germany           & \checkmark~\cite{germansurv}   &             &             &                  \\ 
%UA Emirates & & & & \checkmark~\cite{uae_surv} \\
%Bahrain & & & & \checkmark~\cite{bahrain_surv} \\
%Australia & \checkmark~\cite{eyeswideopen} & &  &\\
%New Zealand & \checkmark~\cite{eyeswideopen} & &  & \\
%Canada & \checkmark~\cite{eyeswideopen} & & & \\
%United States & \checkmark~\cite{eyeswideopen} & & & \\
%Great Britain & \checkmark~\cite{eyeswideopen} & & & \\
%Poland            & \checkmark~\cite{francesurv2}      &         &  \checkmark~\cite{francesurv2}      &      \\ 
%Hungary            & \checkmark~\cite{francesurv2}        &             & \checkmark~\cite{francesurv2} & \\ 
%Ukraine            & \checkmark~\cite{francesurv2}    & \checkmark~\cite{russiasurv, russiasurv2}    & &  \\ 
%Belarus            & \checkmark~\cite{francesurv2}    & \checkmark~\cite{russiasurv, russiasurv2}    & &  \\ 
%Kyrgyzstan            & \checkmark~\cite{francesurv2}    & \checkmark~\cite{russiasurv, russiasurv2}    & &  \\ 
%Kazakhstan            & \checkmark~\cite{francesurv2}    & \checkmark~\cite{russiasurv, russiasurv2}    & &  \\ 
%Russia            & \checkmark~\cite{francesurv2}    & \checkmark~\cite{russiasurv, russiasurv2}    & &  \\ \hline
%....              &      &        &   \\    \hline
%\end{tabular}
%}
%\end{small}
%\caption{Some countries that actively conduct surveillance.}
%\label{surv_table}
%\end{table}

%A number of other countries are passing laws to facilitate mass surveillance.  These laws have differing levels of intensity, which can be seen in Table \ref{surv_table}; the countries with the least intense surveillance laws are listed at the top of the table, and those with the more intense laws are listed at the bottom.  These countries, along with the ``Five Eyes'' participants should be flagged when characterizing transnational detours in the following section.  Another group of countries includes those that block or interfere with content; these are countries that are infamous for censorship: Iran, China, and Russia.  These countries should also be flagged when analyzing transnational detours, as these countries are also unfavorable transit countries due to their practices.  

%\paragraph{France.} France recently passed a new surveillance law that authorizes the government closely monitor the mobile phone and Internet communications of French citizens.  Additionally, the law requires ISPs to install ``black boxes'' that are designed to collect and analyze metadata on the Internet usage of millions of people.  The law allows surveillance without much oversight and the conditions under which the law's powers can be used are vague~\cite{francesurv}. The French Parliament has also made it easier for the government to access encrypted data for criminal investigations by enhancing penalties against companies that refuse to cooperate~\cite{francesurv2}.

%\paragraph{Germany.} A new data retention law was recently passed in Germany that will allow law enforcement agencies to access metadata of phone calls and Internet connections~\cite{germansurv}.  Prior to the bill, telecom providers only retained data for business purposes, but the bill will modify the German Telecommunications Act and require telecom providers to retain traffic data on phone calls and Internet connections.  The data collected includes IP addresses of users as well as the date and time of Internet connections.

%\paragraph{Poland.}  Poland now has stricter surveillance laws by ammending the Police Act.  This gives Polish authorities access to metadata without court approval~\cite{francesurv2}.

%\paragraph{Hungary.}  Hungary has pushed surveillance even further by proposing a new bill that makes it a crime for service providers to use encryption-based application or software.  It also forces ISPs to build back doors that give the government access to data.  Under the bill, ISPs are required to collect metadata on anyone who uses encryption, and it criminalizes the refusal of companies to give collected data to the government~\cite{francesurv2}.

%\paragraph{Russia.} Russia has had a surveillance infrastructure for years: SORM (System of Operative-Investigative Measures).  SORM involves a series of black boxes that ISPs must install; the providers are required to pay for the SORM equipment and installation and they are denied access to the box~\cite{russiasurv}.  If an ISP fails to cooperate, it is fined, and if the violation persists, then its license may be revoked.  Former Soviet Republics have followed in Russia's footsteps: Belarus, Ukraine, and Kyrgyzstan have adopted national interception systems modeled after SORM~\cite{russiasurv}.  Many popular sites, such as Facebook, Google, and Twitter are not hosted in Russia, which challenges surveillance.  In the past year, Russia has passed a data localization law.  This law states that any company that collects personal information from Russian users must stor their data on servers within the country - the main targets of the law were Facebook, Google, and Twitter~\cite{russiasurv2}. 
