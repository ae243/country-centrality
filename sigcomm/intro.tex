\section{Introduction}
\label{intro}

When Internet traffic enters a country, it becomes subject to that country's
laws and policies.  As a result, users, ISPs, and governments have more need
than ever to determine---and control---which countries their traffic is
traversing.  Discovering which countries an end-to-end path traverses and
providing mechanisms to avoid certain countries may help users avoid the
surveillance practices and privacy laws of particular countries; in some
cases, avoiding certain countries may also lower costs or improve performance,
where technologies that certain countries use (\ie, firewalls, traffic
shapers) throttle network traffic speeds. 

An increasing number of countries have passed laws that facilitate mass
surveillance of networks within their territory~\cite{france_surveillance,
netherlands_surveillance, kazak_surveillance, uk_bill}. Governments and
citizens alike may want to divert their Internet traffic from countries that
perform surveillance (notably, the United States~\cite{russia_secure_internet,
routing_errors, dte}).   Additionally, previous work has shown that tromboning
paths---paths that start and end in the same country, but also traverse a
foreign country--- are common~\cite{shah2015characterizing, gupta2014peering};
both users and ISPs may wish to prevent these international detours for
performance and cost reasons.

Defending against these activities requires not only encryption, but
also mechanisms for controlling where traffic goes in the first place:
end-to-end 
encryption conceals some information content, but it does not protect
all sensitive information.  First, many websites do not fully support
encrypted browsing by default; a recent study showed that more than 85\% of
the most popular health, news, and shopping sites do not encrypt by
default~\cite{what_isps_can_see}; migrating a website to HTTPS can be challenging,
and doing so requires all third-party domains on the site (including
advertisers) to use HTTPS.  Second, even encrypted traffic may still reveal a
lot about user behavior: the presence of any communication at all may be
revealing, and website fingerprinting can reveal information about content
merely based on the size, content, and location of third-party resources that
a client loads~\cite{Johnson2013a}. DNS traffic is also revealing and is
almost never encrypted~\cite{what_isps_can_see}.  Third, ISPs often
terminate TLS connections, conducting man-in-the-middle attacks on encrypted
traffic for network management purposes~\cite{mitm_isp}.  And, of course,
encryption offers no solution to interference, degradation, or blocking of
traffic that a country might perform on traffic that crosses its borders.
Finally, a nation-state may collect and store encrypted traffic; if the
encryption is broken in the future, a nation-state may be able to discover the
contents of previous communications.

In this paper, we study two questions: (1)~Which countries do {\em   default}
Internet routing paths traverse?; (2)~What methods can  help governments and
citizens better control transnational Internet paths?  We {\em actively
measure} the paths originating in twenty countries to the most popular
websites in each of these respective countries.
Our analysis in this paper focuses on five countries---Brazil, Netherlands,
Kenya, India, and the United States---for a variety of reasons.  For example,
Brazil has made a concerted effort to avoid traversing certain countries such
as the United States through extensive buildout of Internet Exchange Points
(IXPs). The Netherlands has one of the world's largest IXPs and relatively
inexpensive hosting. Kenya is one of the most well-connected African
countries, but it is still thought to rely on connectivity through Europe and
North America for many destinations, even content that might otherwise be
local (\eg, local newspapers). We highlight many trends that are common across all
of the countries we study; we have also released detailed statistics on all twenty
countries that we measure on the project website and intend to update these on a
periodic basis.

In contrast to all previous work in this area, we measure router-level
forwarding paths, as opposed to analyzing Border Gateway Protocol (BGP)
routes~\cite{karlin2009nation,shah2015characterizing}, which can provide at
best only an indirect estimate of country-level paths to popular sites.
Although BGP routing can offer some information about paths, it does not
necessarily reflect the path that traffic actually takes, and it only provides
AS-level granularity, which is often too coarse to make strong statements
about which countries that traffic is traversing.  In contrast, we measure
routes from RIPE Atlas probes~\cite{ripe_atlas} in each country to the Alexa
Top~1000 domains for each country; we directly measure the paths not only to
the websites corresponding to themselves, but also to the sites hosting any
third-party content on each of these sites.

Even with the benefit of direct measurements, determining which countries a
client's traffic traverses is challenging, for several reasons.  First,
performing direct measurements is more costly than passive analysis of BGP
routing tables; RIPE Atlas, in particular, limits the rate at which one can
perform measurements.  As a result, we had to be strategic about the origins
and destinations that we selected for our study. We study twenty
geographically diverse countries,  focusing on countries in each region that
are making active attempts to thwart transnational Internet paths.  Second, IP
geolocation---the process of determining the geographic location of an IP
address---is notoriously challenging, particularly for IP addresses that
represent Internet infrastructure, rather than end-hosts. We cope with this
inaccuracy by making conservative estimates of the extent of routing detours,
and by recognizing that our goal is not to pinpoint a precise location for an
IP address as much as to achieve accurate reports of {\em significant} off-path 
detours to certain countries or regions. (Section~\ref{datasets} explains
our method in more detail; we also explicitly highlight ambiguities in our
results.) Finally, the asymmetry of Internet paths can also make it difficult
to analyze the countries that traffic traverses on the reverse path from
server to client; our study finds that country-level paths are often
asymmetric, and, as such, our findings represent a lower bound on
transnational routing detours.

We first {\em characterize the current state of transnational Internet
routing detours} (Section~\ref{datasets}).  We explore hosting diversity
and find that only about half of the Alexa Top 100 domains in the five
countries studied are hosted in more than one country; in many cases,
that country is one that clients may want to avoid. Second, even if
hosting diversity can be improved, routing can still force traffic
through a small collection of countries (often surveillance
states). Despite strong efforts made by some countries to ensure their
traffic does not transit certain countries~\cite{brazil_history,
  brazil_break_from_US, brazil_conference, brazil_conference2,
  brazil_human_rights}, their traffic still does so.  For example, over 50\% of the
top domains in Brazil and India are hosted in the United States, and
over 50\% of the paths from the Netherlands to the top domains transit
the United States.  About half of Kenyan paths to the top domains
traverse the United States and Great Britain (but the same half does not
traverse both countries).  Much of this phenomenon is due to
``tromboning'', whereby an Internet path starts and ends in the same country,
yet transits an intermediate country; for example, about 13\% of the
paths that we explored from Brazil tromboned through the United States.
Infrastructure building alone is not enough. ISPs in respective regions
need better encouragements to interconnect with one another to ensure
that local traffic stays local.

Next, we {explore the extent to which a network of overlay relays could help
clients avoid certain countries to popular destinations}
(Section~\ref{avoid_results}). Our results demonstrate that this technique can
be effective for clients in certain countries; of course, the effectiveness of
this approach naturally depends on where content is hosted for that country
and the diversity of Internet paths between ISPs in that country and the
respective hosting sites. For example, our results show that clients in Brazil
can completely avoid Spain, Italy, France, Great Britain, Argentina, and
Ireland (among others), even though the default paths to many popular
Brazilian sites traverse these countries. We also find that some of the most
prominent surveillance states are also some of the least avoidable countries.
For example, many countries depend on ISPs in the United States, a known
surveillance state, for connectivity to popular sites and content.
Additionally, overlay network relays can increase performance by keeping local
traffic local: by using relays in the client's country, fewer paths trombone
out of the client's country.

Finally, we {\em design, implement, and deploy \system{}, a system that allows
a client to access web content while avoiding the traversal of a specified
country} (Section~\ref{system_design}).  We implemented \system{} for
end-users, 
but ISPs could also deploy \system{} proxies to provide country
avoidance as a service to its customers.  \system{} uses a series of overlay
network relays to automatically route a client's traffic around a specified
country.  We evaluate \system{} to assess its ability to avoid certain
countries, as well as the effect of the system on end-to-end performance. We
also discuss the usability and and scalability of the system.  Our evaluation
shows that \system{} can effectively avoid many different countries and
introduces minimal performance overhead. 
