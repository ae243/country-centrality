\section{Discussion}
\label{sec:discussion}

\paragraph{Avoiding multiple countries.} 
We have studied only the extent to which Internet paths can be
engineered to avoid a {single} country.  Yet, avoiding a single country
may force an Internet path into {\em other} unfavorable
jurisdictions. Future work should
explore the feasibility of avoiding multiple countries or perhaps even entire regions.


\paragraph{Evolution over time.}
Our study is based on a snapshot of paths. Over time, paths
change, hosting locations change, IXPs are built, submarine cables are
laid, and surveillance states change.  We are continuing to collect
the measurements that we have presented in this paper to facilitate future exploration
of how these characteristics evolve over time.

%\paragraph{Isolating DNS diversity vs. path diversity.}
%In our experiments, the overlay network relays perform DNS lookups from
%geographically diverse locations, which provides some level of DNS
%diversity in addition to the path diversity that the relays inherently
%provide. This approach somewhat conflates the benefits of DNS diversity
%with the benefits of path diversity and in practice may increase
%clients' vulnerability to surveillance, since each relay is performing
%DNS lookups on each client's behalf. We plan to conduct additional
%experiments where the client relies on its local DNS resolver to map
%domains to IP addresses, as opposed to relying on the relays for both
%DNS resolution and routing diversity.

\paragraph{ISPs controlling country avoidance.} 
Future work includes modifying \system{} to be implemented within an 
ISP.  Adding country avoidance functionality within ISPs 
(government-controlled or otherwise) allows ISPs to provide this as a transparent
service to customers.  A government that wishes to control which countries
its citizens' traffic is traversing might deploy \system{} in the country's ISPs.

\paragraph{Additional \system{} features.}  
The oracle could add additional steps in the decision
chain introduced in Section \ref{multiplex} that take into account
relay and path loads.  For example, if multiple relays provide a path
to a domain that does not traverse the specified country, then the
decision between the suitable proxies could be determined based on
current relay load or performance.  Our current implementation of
\system{} re-computes all paths once per five days; we could only
re-compute paths when necessary.  For example, a BGP monitoring system
detect routing changes and trigger path measurements.

%Additional features can be
%implemented at the relay to help preserve client privacy.  An example
%would be to use the relay as a mix, or to send out fake traffic to
%confuse an attacker that may be trying to perform traffic analysis at
%the relay.
